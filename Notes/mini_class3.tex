\documentclass[11pt]{article}

\oddsidemargin=0.25truein \evensidemargin=0.25truein
\topmargin=-0.5truein \textwidth=6.0truein \textheight=8.75truein

%\RequirePackage{graphicx}
\usepackage{comment}
\usepackage{hyperref}
\urlstyle{rm}   % change fonts for url's (from Chad Jones)
\hypersetup{
    colorlinks=true,        % kills boxes
    allcolors=blue,
    pdfsubject={ECON-UB ???233, Macroeconomic foundations for asset pricing},
    pdfauthor={Dave Backus @ NYU},
    pdfstartview={FitH},
    pdfpagemode={UseNone},
%    pdfnewwindow=true,      % links in new window
%    linkcolor=blue,         % color of internal links
%    citecolor=blue,         % color of links to bibliography
%    filecolor=blue,         % color of file links
%    urlcolor=blue           % color of external links
% see:  http://www.tug.org/applications/hyperref/manual.html
}

%\renewcommand{\thefootnote}{\fnsymbol{footnote}}

% section spacing and fonts
\usepackage[small,compact]{titlesec}

% list spacing
\usepackage{enumitem}
\setitemize{leftmargin=*, topsep=0pt}
\setenumerate{leftmargin=*, topsep=0pt}

% attach files to the pdf
\usepackage{attachfile}
    \attachfilesetup{color=0.75 0 0.75}

\usepackage{needspace}
% example:  \needspace{4\baselineskip} makes sure we have four lines available before pagebreak

\usepackage{verbatim}


% document starts here
\begin{document}
\parskip=\bigskipamount
\parindent=0.0in
\thispagestyle{empty}
{\large Data Bootcamp @ NYU Stern \hfill Dave Backus \& Glenn Okun}


\bigskip\bigskip
\centerline{\Large \bf Data Bootcamp:  Class \#3}
\centerline{Revised: \today}

\begin{comment}
* Today:  control, data frames 
* Before we start
    - start Spyder
    - download bootcamp_control.py
\end{comment}


\section{Course overview}

\begin{itemize}
\item Objective:  Learn enough about Python to do useful things with data.
\item Target audience:  Programming newbies.
Anyone can do this with a little persistence and the help of friends.
\item Trigger warning:  This will take some effort.  But it's worth it.

\item {\bf Work habits:}
\begin{itemize}
\item Advice.  This will seem mysterious at first, but if you stick with it,
you'll find it starts to look familiar, even make sense.
\item Practice.  Any time you have something to do with data, try it out in Python.
Play around.  Have fun!
\item Friends.  Work with friends, and make new friends who know how to code.
\item Help.  If you get stuck, ask for help --- from friends,
from the Bootcamp group (post a problem), Stack Overflow, etc.
\end{itemize}
\item Team:  Dave Backus, Glenn Okun, Sarah Beckett-Hile, and a rotating group of ninjas.

\item Resources:
\begin{itemize}
\item Bootcamp Group:  \url{https://groups.google.com/forum/#!forum/nyu_data_bootcamp}.
Post comments and questions here.

\item GitHub repository:  \url{https://github.com/DaveBackus/Data_Bootcamp}.
All the docs and programs are here.  This document is in the Notes folder.

\item This document:  The online version comes with links.
\end{itemize}

\end{itemize}


\section{Overall plan}

\begin{itemize}
\item First class:  Python basics, examples.
\item Today's class:  graphics.
\item Next (and last class) in two weeks:  data management, SQL
\item {\bf It's possible we'll run a more formal course in January}
\end{itemize}


\section{Today's plan}

\begin{itemize}
\item Theme:  graphics.
\item Standard ``scientific'' packages.
\item Reminders.
\item Graphics with Matplotlib: two routes, one destination.
\item Exercise review (Sarah).
\end{itemize}


\section{Standard packages}

Python is not just a program, it's an open source collection of tools
that includes both basic
Python and a collection of packages written by different people.
The good news is that you can often find a package to do whatever it is
you want to do.
The bad news is that there's not necessarily any uniformity or quality control.
The standard packages --- which we'll talk more about in a minute ---
are well written, well documented, and have armies of users who spot and correct
problems.
Some of the others less so.
Our suggestion is to stick to the mainstream packages, such as those in the
\href{http://docs.continuum.io/anaconda/pkg-docs.html}{Anaconda distribution},
until you're reasonably comfortable with Python.


Some of the leading mainstream packages for numerical (``scientific'') computation ---
 all of which are in the Anaconda --- are
\begin{itemize}
\item \href{http://www.numpy.org/}{NumPy}.
Tools for numerical computing, including linear algebra.
Compare this to Excel, where the basic unit is a cell, a single number.
In NumPy the basic unit is a vector (a column) or matrix (a table).
In this respect, it's a direct competitor to Matlab.
We won't see much of NumPy here, but it's the foundation for Pandas.

\item \href{http://matplotlib.org/index.html}{Matplotlib}.
The leading graphics library for Python and our focus today.

\item \href{http://pandas.pydata.org/}{Pandas}.
Tools for managing and manipulating data.
The basic unit is a DataFrame, which is a table of data with labels
for observations and variables.
If you've used R, you'll be familiar with the idea.
If not, you'll be familiar with it in a couple weeks.
We'll touch on this today and spend a fair amount of time with it at
our next meeting.
\end{itemize}

We love the term ``wrapper,'' it makes it sound like we know what we're doing.
It's an informal programming term that means a simplified interface
for more basic programming functions. Cool, eh?
Here are some popular wrappers for Matplotlib:
\begin{itemize}
\item \href{http://matplotlib.org/users/pyplot_tutorial.html}{Pyplot}.
This has two interfaces for Matplotlib,
a simple one similar to Matlab and a more complicated object-oriented one
(which means we'll use objects and methods).
We'll describe both.

\item \href{http://web.stanford.edu/~mwaskom/software/seaborn/}{Seaborn}.
Another interface for Matplotlib.  Not part of the Anaconda distribution,
but it seems to us to have some potential.
Take a look at the examples, see what you think.

\item \href{http://wiki.scipy.org/PyLab}{PyLab}.
A combination of (basically) NumPy and Pyplot designed to replace Matlab.
Convenient if you're familiar with Matlab, probably not otherwise.
The
\href{http://matplotlib.org/dev/faq/usage_faq.html#matplotlib-pylab-and-pyplot-how-are-they-related}
{Matplotlib FAQ}
adds:  ``it is no longer recommended.''

\end{itemize}
We'll use Pyplot today.
And Sarah may use Seaborn when she goes through last week's exercise.


\section{Reminders}

\begin{itemize}
\item Environment:  Python 3.4 in Spyder.
\item Assignments, strings, lists, slicing:
\begin{verbatim}
x, y, z = 2*3, 2**3, 2/3    # yes, three at once!
a, b = 'some', 'thing'
c = a + b

numbers = [x, y, z]
strings = [a, b, c]
all = numbers + strings
print(all)
print([type(all), all[:3], all[3:])
\end{verbatim}

\item Objects and methods.  Most of the things we've seen are objects:  {\tt x, a, all}, etc.
They have methods defined for them -- useful things you can do with them.
For more on the methods available for the object {\tt all},
go to the IPython console and type {\tt all.[tab]}.
We call this ``tab completion.''
For more on a specific method called {\tt foo},
go to the Object explorer and type {\tt all.foo}.

\item Reading csv files.
This was part of last week's assignment:
\begin{verbatim}
import pandas as pd
url1 = 'https://raw.githubusercontent.com/fivethirtyeight/data/master/'
url2 = 'college-majors/recent-grads.csv'
df = pd.read_csv(url1+url2)
\end{verbatim}
What is {\tt df}?  What's in it?
\begin{verbatim}
properties = [type(df), df.columns, df.index, df.shape, df.head()]
for property in properties :
    print(property, end='\n\n')
\end{verbatim}
Use tab completion to find more.


\item {\bf Exercises.}
\begin{itemize}
\item How do we find the first item in the list {\tt all}?  The last?
\item What is {\tt len(all)}?
\item How do we find the third column name in {\tt df}?  What is it?
\end{itemize}

\end{itemize}


\section{Graphics overview}

Graphics are horrendously complicated.  Programs like Excel
do their best to hide this fact, but if you ever try to customize
a chart it quickly rears its ugly head.
A graph might include, for example:  the type (line, bar, etc);
the color and thickness of lines, bars, or markers;
title and axis labels;
their location, fonts, and font sizes;
tick marks (location, size);
background color;
and so on.

Matplotlib gives you control over all of these things and more ---
but it's complicated, there are lots of moving parts.
The set of defaults is collected in what they call the
{\tt matplotlibrc} file, and
\href{http://matplotlib.org/1.4.0/users/customizing.html}{it's enormous}.
That's why people have come up with simpler interfaces,
Pyplot and Seaborn among them.

Bottom line:  Graphics are complicated, but that's the nature of the beast,
not something specific to Python.
What Python gives you is total control.
As always, you should start simple, work up to more complicated things gradually.


\section{Pyplot's simple interface}

\begin{itemize}
\item Overview.  Pyplot is a module of Matplotlib.
Here we'll use its simple interface.
It looks like Matlab, but if that means nothing to you forget we said it.
Either way, it's a good place to start.

\item References:
\begin{itemize}
\item Matplotlib's
\href{http://matplotlib.org/users/pyplot_tutorial.html}{Pyplot tutorial}.
(This is excellent, check it out.)
\item Matplotlib's
\href{http://matplotlib.org/gallery.html}{gallery of examples}.
Graphs along with the code that produced them, a great place to start.
\item Sargent and Stachurski's ``Quantitative Economics,''
the section on
\href{http://quant-econ.net/py/matplotlib.html}{Matplotlib}.
Good, but a little terse for our taste.
\end{itemize}

\item Line plots.
Here we take 11 years of annual US GDP data and plot it against time.
Also consumption (``personal consumption expenditures''), so that we can
show how to add a second line to a graph.
We put all of these things, plus time, in lists and go to work:
\begin{verbatim}
import matplotlib.pyplot as plt     # import pyplot module

# data
gdp  = [13271.1, 13773.5, 14234.2, 14613.8, 14873.7, 14830.4, 14418.7,
        14783.8, 15020.6, 15369.2, 15710.3]
pce  = [8867.6, 9208.2, 9531.8, 9821.7, 10041.6, 10007.2, 9847.0, 10036.3,
        10263.5, 10449.7, 10699.7]
year = [2003, 2004, 2005, 2006, 2007, 2008, 2009, 2010, 2011, 2012, 2013]

# plot
plt.plot(year, gdp)         #  plot(x,y)
\end{verbatim}
We can do lots of things to dress up the plot.
Here are some examples, but find your own with {\tt plt.[tab]}:
\begin{verbatim}
plt.plot(year, pce, color='magenta', linewidth=2)  # add another line
plt.title('US Real GDP and Consumption')    # title
plt.ylabel('Billions of 2009 USD')          # y axis label
plt.legend(('GDP', 'Consumption'), loc=0)   # legend
plt.text(2010, 14000, 'GDP')                # add text at specific location
plt.text(2010, 9200, 'Consumption')         # same again
plt.ylim((0, 16000))                        # change y axis limits
plt.tick_params(length=6, width=2, colors='red')    # yikes!
plt.savefig('bootcamp_test.pdf')            # export as (in this case) pdf
plt.show()                                  # close plot
\end{verbatim}

\item {\bf Exercise.} Run the basic plot GDP on your computer and
\begin{itemize}
\item Change the color of the GDP line to green.
\item Add a marker.
\item Left-justify the title.
\item Extra credit.  Change the title's font size to 20.
\item More extra credit.  What does the {\tt plot} argument {\tt alpha=0.5} do?
\end{itemize}


\item Bar charts.
We'll work with two examples here.
The first is to express our line plot of US GDP as a bar chart.
We do that with the command:
\begin{verbatim}
plt.bar(year, gdp)                          # bar(x,y)
\end{verbatim}
This isn't the most attractive bar chart in the world, but it works.
Here's an example that looks better:
\begin{verbatim}
plt.bar(year, gdp, width=0.9, color='blue', align='center', alpha=0.7)
plt.show()
\end{verbatim}


Here's another example, a standard chart in the MBA Global Economy class.
The difference here is that one axis is a country, which is not numerical data.
(There's a similar
\href{http://matplotlib.org/examples/lines_bars_and_markers/barh_demo.html}{example}
in the Matplotlib gallery.)
Here's the data:
\begin{verbatim}
codes     = ['USA', 'FRA', 'JPN', 'CHN', 'IND', 'BRA', 'MEX']
countries = ['United States', 'France', 'Japan', 'China', 'India',
             'Brazil', 'Mexico']
gdppc     = [53.1, 36.9, 36.3, 11.9, 5.4, 15.0, 16.5]
\end{verbatim}
Now think about constructing a bar chart.
The Pyplot command is {\tt bar},
which works just like {\tt plot}:  the first two arguments
are the x and y axes.
Here there's no obvious x axis --- the country codes and names won't work ---
so we need to make one.
These two lines (i)~make up a counter of the same length as {\tt gdppc}
that we can use
and (ii)~generate a bar chart.
\begin{verbatim}
other_axis = range(len(gdppc))
print(other_axis)
plt.bar(other_axis, gdppc, align='center')
\end{verbatim}
This isn't the most elegant chart you'll see, but it's functional.

You might stop here and think:  What would you do to dress this chart up,
make it more attractive and compelling?

Here are some examples that might get you started:
\begin{verbatim}
plt.xticks(other_axis, codes, fontsize=12)
plt.xlim((-0.75, 6.75))
plt.ylabel('GDP Per Capita (thousands of USD)')
plt.title('GDP Per Capita', fontsize=16, loc='left')
\end{verbatim}
What do you think?  What would you add?

\item Horizontal bar charts.
The same chart looks better if we rotate it.
That's easily done by replacing the command {\tt bar} with {\tt barh}.
Even better, we now have room to add the country names, not just the three-letter
country codes.
\begin{verbatim}
plt.barh(other_axis, gdppc, align='center', alpha=0.7)
plt.ylim((-0.6, 6.6))
plt.yticks(other_axis, countries, fontsize=14)
\end{verbatim}

\end{itemize}



\section{Pyplot's object-oriented interface}

\begin{itemize}
\item Overview.
This is close to classic Matplotlib,
which is said to give you greater flexibility and control.
You'll have to decide for yourself whether it's worth it.
Eventually yes, but it probably makes sense to start with simple interface.
We'll go through this at high speed to give you a sense
of the syntax.

\item References:
\begin{itemize}
\item \href{http://matplotlib.org/index.html}{Matplotlib}
doesn't have a tutorial.
The documentation sends you to the gallery of examples and outside sources.
\item Sargent and Stachurski's ``Quantitative Economics'' is pretty good, if a little terse.
See the section on
\href{http://quant-econ.net/py/matplotlib.html}{Matplotlib}.
\end{itemize}

\item Basic syntax.  A typical graph has commands like these:
\begin{verbatim}
import matplotlib.pyplot as plt
fig, ax = plt.subplot()
\end{verbatim}
{\tt fig} and {\tt ax} here are variable names;
we can call them anything we want, but these are standard.
{\tt fig} is referred to as a figure object, it controls the overall figure area.
If we stop here, we just get a blank box.
{\tt ax} is referred to as an axis object,
 it controls the content of the figure.
 We'll give some examples below.

\item Line plots.  We'll start with the same data we used earlier:
\begin{verbatim}
gdp  = [13271.1, 13773.5, 14234.2, 14613.8, 14873.7, 14830.4, 14418.7,
        14783.8, 15020.6, 15369.2, 15710.3]
pce  = [8867.6, 9208.2, 9531.8, 9821.7, 10041.6, 10007.2, 9847.0, 10036.3,
        10263.5, 10449.7, 10699.7]
year = [2003, 2004, 2005, 2006, 2007, 2008, 2009, 2010, 2011, 2012, 2013]
\end{verbatim}
To generate a plot of US real GDP, we apply the method {\tt plot}
to {\tt ax} much as we did before:
\begin{verbatim}
fig, ax = plt.subplots()                    # sets up single plot
ax.plot(year, gdp, color='blue', linewidth=2, alpha=0.8)
plt.show()                                  # closes plot
\end{verbatim}

\item Multiple plots.
\begin{verbatim}
fig, ax = plt.subplots(nrows=2, ncols=1, sharex=True)
# first figure
ax[0].plot(year, gdp, color='blue', linewidth=2, alpha=0.8)
ax[0].text(2010, 14000, 'GDP')
# second figure
ax[1].plot(year, pce, color='magenta', linewidth=2, alpha=0.8)
ax[1].text(2010, 9200, 'Consumption')
plt.show()
\end{verbatim}

\item {\bf Exercise.}
Convert the double line plot above to bar charts.


\end{itemize}



\section{Before the next class}

{\bf Exercise.}
Randy Olson has a
\href{http://www.randalolson.com/2014/06/28/how-to-make-beautiful-data-visualizations-in-python-with-matplotlib/}
{nice blog post} (search ``randy olson beautiful'')
about creating effective graphics.
He starts with this gif:
\url{http://gfycat.com/ImprobableFemaleBasenji}.
Skim his post, play the gif, and
%
\begin{itemize}
\item  Write a Python script that recreates the data in the figure,
both numbers and labels.
(There are only a few points, you can just type them into lists.)
\item Approximate the various graphs using Matplotlib.
\item Play around with the graphics parameters.
What is the best graph you can produce?
What do you like about it?
\item Variant:  Do the same with our GDP per capita bar charts.
\end{itemize}


\section*{Today's code}

Attached.  Download this pdf file, open in Adobe Acrobat or the equivalent,
and click on the pushpins:
\attachfile{../Code/Python/bootcamp_graphics.py}

Also here:
{\small
\verbatiminput{../Code/Python/bootcamp_graphics.py}
}


\end{document}


\section{Python basics}

\begin{itemize}
\item Environments:  Python 3.4 in Spyder, editor, IPython console.
\item Reference:
{Python tutorial},
\href{https://docs.python.org/3.4/tutorial/introduction.html}{Section 3}.

\item Operations:  \verb: 2*3, 2 * 3, 2/3, 2^3, 2**3, log(3), etc :
% http://en.wikibooks.org/wiki/LaTeX/Paragraph_Formatting#Verbatim_text

\item Assignments:  {\tt x = 2, y = 3.5, z = x/y, etc }
\item Comments and {\tt print}:  \verb:#, """, #%%, print(x), etc. :
\item Strings:  {\tt a = 'some', b = 'thing', c = a+b, d = '11.32', etc.}

\item Two commands,  {\tt len} and {\tt type}:
{\tt len(first), len(x), type(first), type(x)}.

\item Slicing.  Find an element of {\tt a}: {\tt a[1], a[0], a[1:3], a[-1], a[:2], a[1] = 'z', etc.}

\item {\bf Exercise.}  Set {\tt first} equal to your first name, {\tt last} equal to your last name.
\begin{enumerate}[label=(\alph*)]
\item Set {\tt full} equal to your first and last names together with a space between them.
\item Set {\tt lastfirst} equal to your last name, comma, space, first name.
\end{enumerate}

\item Objects and methods.
Most things in Python are {\bf objects}
(for example, {\tt x, z, first}).
{\bf Methods} are ready-to-go things we can do with them.
The available methods depend on the object.
To get a list of available methods for a given object,
type in the IPython console: {\tt object.[tab]}.
When you find a method you're interested in,
go to Spyder's Object explorer
and type {\tt object.method} for a description and syntax.
Examples:  \verb:first.find('a'), x.is_integer():.

\item {\bf Exercise.}  Take {\tt lastfirst} and see what methods are available.
\begin{enumerate}[label=(\alph*)]
\item Use {\tt lastfirst.find} to find the location of the comma and
decompose {\tt lastfirst} into {\tt first} and {\tt last}.
\item Find a method that converts {\tt last} to all lower case.
\end{enumerate}

\item Lists: collections of things, possibly different, defined by square brackets
{\tt []} with commas between items.
Slicing conventions similar to strings.
Examples:  {\tt numbers = [x, y, z], type(numbers),
print(numbers),
strings = [a, b, c, d],
all = numbers + strings, type(all),
print(all),
all[0], all[-1], all[3:], etc.}

\end{itemize}


\section{Inputting data}
\begin{itemize}
\item Packages:  Collections of commands that do more than basic Python.
There are packages that do almost anything, but we'll focus on a couple:
Matplotlib and pandas.  Matplotlib is a graphics package, we'll talk about it
another time.
We touch on pandas --- briefly --- below.

In Python --- and many other languages --- we need to tell Python to make them available.
Here are some options for doing this with a mythical package {\tt xyz}:
\begin{itemize}
\item {\tt from xyz import *}.
This imports all the commands from the package {\tt xyz}.
You can replace the star
with specific commands and import only them.
A command {\tt foo} is then executed by typing {\tt foo}.
\item {\tt import xyz}.  This imports the whole thing.
A command {\tt foo} is then executed by adding {\tt xyz.}
(note the period) before its name:  {\tt xyz.foo}.
\item {\tt import xyz as x} is the most common syntax.
It allows the shorthand {\tt x.foo}.
We'll do this momentarily with the package {\tt pandas}.
\end{itemize}
For more, see the
\href{https://docs.python.org/3.4/tutorial/modules.html}{Section 6}
of the Python tutorial.
This refers to modules, which could be packages, subsets of packages, or even
code you've written yourself.

\item The pandas package:  pandas = (Python Data Analysis Library).
This is the central package for data work,
it allows you to use Python for data management along similar
lines to the program R.
We'll import it typically with the command {\it import pandas as pd}
and execute commands with syntax of the form {\t pd.command}.

\item Reference:  pandas has the usual high-quality
{documentation}
we expect from Python packages.
Search ``python pandas'' or go to
\url{http://pandas.pydata.org/}.
We'll focus here on the input-output tools:
\url{http://pandas.pydata.org/pandas-docs/stable/io.html}.

\item Reading csv files.  We've found that csv files
are the most useful data format, far better than xls or xlsx.
Its simple structure (entries separated by commas or similar) allows easy and rapid input.
That's a general statement, not a statement about Python.

Suppose you have (as we do) a csv file in a parallel directory {\tt Data}.
Then we can read it with the \verb:read_csv: command in pandas:
\begin{verbatim}
import pandas as pd
file = '../Data/test1.csv'
df = pd.read_csv(file)
\end{verbatim}
This reads the file {\tt test1.csv} into {\tt df}.
If you try {\tt type(df)} you'll find that {\tt df} is a DataFrame,
the standard object type in pandas.
Some of the methods you could try:
{\tt df.columns} (the column labels),
{\tt df.index} (the row labels),
{\tt df.head(2)} (the top two rows),
and {\tt df.tail(2)} (the bottom two rows).
There are only three rows in this one, so you could simply
print the whole thing, but with large data sets this is very useful.

{\bf Comment.}
The tricky part of this is setting the working directory:
where Python looks for things.
More on that another time.
% ??


\item Reading csv's from url's.
Even better, {\tt read\_csv} also reads from url's.
The same file is in our GitHub repository (talk about how to find it,
and the Raw button).
We can read it with
\begin{verbatim}
import pandas as pd     # this is redundant if we've already imported it
url = 'https://raw.githubusercontent.com/DaveBackus/Data_Bootcamp/master/Code/Data/test1.csv'
df = pd.read_csv(url)
\end{verbatim}
Or we could read the World Economic Outlook (WEO) database from the IMF.
This reads in the whole thing:
\begin{verbatim}
url = 'http://www.imf.org/external/pubs/ft/weo/2014/01/weodata/WEOApr2014all.xls'
weo = pd.read_csv(url, sep='\t')    # tab = \t
\end{verbatim}
%{\bf Exercise.}
Try this and list the contents.
What does the file look like?

\item Spreadsheets.  They have a more complex structure and the possibility
of more than one sheet.
Here's an example that reads in the first sheet:
\begin{verbatim}
file = '../Data/test2.xlsx'
xls = pd.read_excel(file)       # default is first sheet
\end{verbatim}
% ***********  ?? more at some point


\item Copying from url's.
Here's an example that copies a file from a url:
\begin{verbatim}
import urllib.request           # this is a module from the package urllib
file = 'foo.csv'
url = 'https://raw.githubusercontent.com/DaveBackus/Data_Bootcamp/master/Code/Data/test1.csv'
urllib.request.urlretrieve(url, file)
\end{verbatim}
This way you can copy the file to your own computer and read it in from there.
This does the same thing, courtesy of Sarah:
\begin{verbatim}
f = urllib.request.urlopen(url)
file = 'foo_sbh.csv'
with open(file, 'wb') as local_file:
    local_file.write(f.read())
\end{verbatim}

\item {\bf Exercise.}  Create a spreadsheet and save it as both csv and xlsx.
Write code to read them into Python.


\item Zip files.  The {\tt zipfile} module (part of basic Python)
lets you do whatever you want with zip files.
Some examples:
\begin{verbatim}
import pandas as pd
import urllib
import zipfile
import os

url  = 'http://databank.worldbank.org/data/download/WDI_csv.zip'
file = os.path.basename(url)            # strip out file name
urllib.request.urlretrieve(url, file)   # copy to disk

# see what's there
print(['Is zipfile?', zipfile.is_zipfile(file)])
zf = zipfile.ZipFile(file, 'r')
print('List of zipfile contents (two versions)')
[print(file) for file in zf.namelist()]
zf.printdir()

# extract a component
csv = zf.extract('WDI_Data.csv')        # copy to disk
df1 = pd.read_csv('WDI_Data.csv')       # read
print(df1.columns)                      # check contents

# alternative:  open and read
csv = zf.open('WDI_Data.csv')
df2 = pd.read_csv(csv)
print(df3.columns)
\end{verbatim}
For more, see the section of the Python tutorial that describes the
\href{https://docs.python.org/3.4/library/zipfile.html#module-zipfile}{zipfile module}.

\end{itemize}

\section{Projects}


It's best to have a goal, something you're working toward.
Might be something you can show potential employers to
demonstrate your skill set.


\section{Examples}

These illustrate some of the possibilities of Python
with data that's available free from online sources.
This goes beyond what we've learned so far, but gives you a sense
of what we'll be doing.
Some examples:
\begin{itemize}
\item FRED.  Great online source of macroeconomic data.
\item Fama-French.  Great source of equity returns on portfolios sorted by type and industry.
Used to illustrate boxplots.
\item World Bank.  Country indicators. Used to illustrate bubble plot (scatter plot with third variable
indicated by size of ``bubble'').
\end{itemize}
See the code for details.
We're still working on Yahoo's Options data,
which went from working to not working in the last week.

What examples would you like to see?
Tell us, or post a suggestion on the Group.


\section{Coming attractions}

In the works:
\begin{itemize}
\item Basic programming tools:  control flow, functions.
\item Data:  the pandas package.
\item Graphics:  the Matplotlib package.
\end{itemize}

If there's interest:
\begin{itemize}
\item SQL databases.
\item Scraping data from websites.
\item Interactive web graphics:  Plot.ly and Google charts.
\item Anything else?
\end{itemize}

\section{Before the next class}

\begin{itemize}
\item Play around with Python.
Play around with some of the code snippets you've seen.
Experiment with new methods.
Try to read data you're using for other purposes.
Run programs you find in the documentation.

\item Think about projects.
Skim through our list of projects and data sources, see if anything crosses your mind.
Or ask your employer.  Or anywhere else you find data.

\item Do the assignment below.  Really, give it a try.  Tell us what undergrad major pays the most.
Think about graphics you can use to make the case.
Bounce around ideas for a (small) project of your own.

\item Get help when you need it.
If you get stuck, ask friends for help, or post a question on the Google Group.
Uee StackOverflow rules if possible:  include a self-contained example of your problem, code that others can fix.

\item {\bf Exercise.}
Read the article from the FiveThirtyEight (first link below) about income
by college major of recent graduates.
It takes data from a survey run by the US government (the American Community Survey) and
produces figures summarizing the salaries of students who majored in
various subjects.  They've kindly posted the data online --- also the R
code --- in the second link.

Your mission:

(a) Read the data in the file {\tt recent-grads.csv} in the 538 GitHub
repository (link below).
You should be able to read it into Python straight from the url.
Make sure you use the link from the Raw file.
How many data points do you have?  How many
variables?

(b) Think about graphs you'd like to see, whether in the article or
not.  You could, for example, look through the Matplotlib gallery
(link below) and see if there are any graph types that you think you
could use to present this data effectively.

(d) If you're feeling brave, produce a graph of your own.  We haven't
talked much about graphics yet, so if you get stuck, ask questions or stop.

{\it Links\/} \\
Article:  \url{http://fivethirtyeight.com/features/the-economic-guide-to-picking-a-college-major/} \\
Code and data:
\url{https://github.com/fivethirtyeight/data/tree/master/college-majors} \\
Gallery:  \url{http://matplotlib.org/1.2.1/gallery.html}


\end{itemize}

%\pagebreak
\section*{Python code}

Attached.  Download this pdf file, open in Adobe Acrobat or the equivalent,
and click on the pushpins:
\attachfile{../Code/Python/bootcamp_basics.py}
\attachfile{../Code/Python/bootcamp_examples.py}

\subsection*{Basics}
{\small
\verbatiminput{../Code/Python/bootcamp_basics.py}
}

\subsection*{Examples}

{\small
\verbatiminput{../Code/Python/bootcamp_examples.py}
}

\end{document}
