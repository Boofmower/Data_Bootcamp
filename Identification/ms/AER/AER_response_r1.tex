\documentclass[11pt,letterpaper]{article}

\oddsidemargin=0.25truein \evensidemargin=0.25truein
\topmargin=-0.5truein \textwidth=6.0truein \textheight=8.75truein

\usepackage{graphicx}
\usepackage{comment}
%\usepackage{booktabs}
\usepackage[small, compact]{titlesec}

% list spacing
\usepackage{enumitem}
\setitemize{leftmargin=*, topsep=0pt}
\setenumerate{leftmargin=*, topsep=0pt}

% for figs and tabs
\usepackage[margin=0pt, labelsep=period, font=large, labelfont=bf]{caption}

% attach files to the pdf
\usepackage{attachfile}
    \attachfilesetup{color=0.75 0 0.75}

\usepackage{needspace}
% example:  \needspace{4\baselineskip} makes sure we have four lines available before pagebreak

\usepackage{hyperref}
\hypersetup{
    letterpaper=true,
    colorlinks=true,        % kills boxes
    allcolors=blue,
    pdfauthor={Backus, Chernov, \& Zin},
    pdfstartview={FitH},
    pdfpagemode={UseNone},
    pdfnewwindow=true,      % links in new window
% see:  http://www.tug.org/applications/hyperref/manual.html
}

\renewcommand{\thefootnote}{\fnsymbol{footnote}}
\newcommand{\var}{\mbox{\it Var\/}}

% document starts here
\begin{document}
\parskip=0.75\bigskipamount
\parindent=0.0in
\thispagestyle{empty}
{\Large\bf General remarks} \\
\today

We appreciate the comments we received on the paper,
they go deeper than anything we've heard to date from others
and led us to change several things in the paper.
What follows is a list of what we understood the reports to say
and how we addressed the concerns they raise.

One issue deserves more attention than we gave it in the paper.
The paper is an attempt to characterize what we regard as a central issue in
macroeconomics:  identification.
We wrote it as simply as we could so that it would appeal to the large audience of applied
macroeconomists as well as the smaller audience of econometricians.

Identification is a central issue throughout economics,
and macroeconomics and macro-finance.
Following influential work by Cochrane and by Singleton and coauthors,
we've heard repeatedly that models like those we describe are not identified.
We think we've done a lot to clarify the issue.
We'd say, instead, that identification requires one or more restrictions
of a specific sort.
Our application to bond pricing models is...



\section*{Importance}

??
\begin{itemize}
\item Big issue.
\item Clean result.
\item ??
\end{itemize}




\end{document}




Marty,

You may recall that  my "Identification" paper with Dave and Stan was rejected last fall.
Which is fine, we're all grownups, we learned something from the process 
and there are lots of other journals out there to try next.  
Our plan, though, was to submit it to the AEJM.
Since they'll see the reports, we thought we should address them.
Once we did that, we thought we'd run our response by you.

The main issue is that Referee \#1 is simply wrong.  It's an honest
mistake, and not an obvious one, but we think it's pretty clear the
report is just wrong.  Which takes the report's conclusion that``the
paper is fundamentally flawed'' off the table.  We also 
think we have addressed the other issues, but that's your
call, not ours.  [more?]

Let me know how you would like to handle this.  We'd be happy to have
another shot at the AER, but if you think that's unlikely to do much
but waste your time and ours, let us know and we'll try the AEJM
instead.

Best,
Mike