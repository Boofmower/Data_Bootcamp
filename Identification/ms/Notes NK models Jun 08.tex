\documentclass[11pt]{article}

\oddsidemargin=0.25truein \evensidemargin=0.25truein
\topmargin=-0.5truein \textwidth=6.0truein \textheight=8.75truein

%\RequirePackage{graphicx} 
%\RequirePackage[hypertex]{hyperref}
\RequirePackage{comment}

\renewcommand{\thefootnote}{\fnsymbol{footnote}}
\newcommand{\var}{\mbox{\it Var\/}}

% document starts here
\begin{document}
\parskip=\bigskipamount
\parindent=0.0in
\thispagestyle{empty}
\begin{flushright} Backus \& Zin \end{flushright}

\bigskip
\centerline{\Large \bf Notes on New Keynesian Models%
\footnote{Working notes, no guarantee of accuracy or sense.}}
\centerline{(Started: June 8, 2006; Revised: \today)}

\bigskip
The idea is to work through some simple models to 
see how they work.  
The emphasis is on their dynamic structure, 
not their economic foundations.   

%\end{document}
\subsection*{Cochrane's example} 

Model.  
This is pretty basic, but illustrates how the model is solved.  
We adapted it from Cochrane's ``identification'' paper.  
Not really a New Keynesian Model, 
more like a Cagan-style model of hyperinflation.
The model has two equations, 
\begin{eqnarray*}
    i_t &=& r + E_t p_{t+1} + e_1^\top x_{t} \\
    i_t &=& r + \tau p_t + e_2^\top x_t ,
\end{eqnarray*}
plus a stationary process for the shocks, 
\begin{eqnarray*}
    x_{t+1} &=& A x_t + B w_{t+1} ,
\end{eqnarray*} 
where $\{ w_t \} \sim \mbox{NID}(0,I)$.
The variables are the nominal interest rate $i$ and inflation $p$.
Think of the first equation as a simple version of an Euler equation (EE) 
and the second as a Taylor rule (TR).
We set $r=0$ to keep things simple.    

Solution.  
The model has two endogenous variables, one static ($i$)
and one dynamic ($p$), terminology that should be clear in a minute.
If we substitute for $i$ (the static endogeneous variable)  
we're left with the expectational difference equation
\begin{eqnarray*}
    E_t p_{t+1} &=& \tau p_t  + (e_2 - e_1)^\top x_{t} .
\end{eqnarray*}
Solution methods are summarized in the Appendix.  
The simplest one is to guess a solution of the form 
$ p_t = a^\top x_t$.
Then $E_t p_{t+1} = a^\top A x_t $.
The equation becomes 
\begin{eqnarray*}
    a^\top A &=& \tau a^\top x_t + (e_2 - e_1) x_t .
\end{eqnarray*}
Collecting coefficients of $x_t$, we have the solution, 
\begin{eqnarray*}
        a^\top &=& (e_2-e_1)^\top (\tau I- A)^{-1} .
\end{eqnarray*}
If we expand the implied geometric series, 
we see this as a linear combination of 
the discounted sum of expected future $x$.


\subsection*{New Keynesian model}

Model.  
Here's another one, a streamlined version of 
Clarida-Gali-Gertler.
[Sorry, I couldn't resist playing with the notation.]  
\begin{eqnarray*}
    i_t &=& \alpha E_t y_{t+1} + E_t p_{t+1}  \\
    p_t &=& \psi_g g_t + \psi_p E_t p_{t+1} + e_p^\top x_t  \\
    i_t &=& \tau_g g_t + \tau_p p_t + e_m^\top x_t \\
    y_{t} &=& a_t + g_{t} \;\;=\;\; e_a^\top x_t + g_t .  
\end{eqnarray*}
The variables are the nominal short rate $i_t$, 
consumption/output growth $y_t$, 
inflation $p_t$, 
growth of the output gap (deviation from optimum) $g_t$, 
and growth of ``full-employment'' output $a_t$.  
The equations are, in order:  
an Euler equation, a Phillips curve, 
a Taylor rule, and the definition of output (=consumption).  

Solution.  
Endogenous variables:  $y, p, i, g$.  
Of these $y$ and $p$ are dynamic forward-looking variables, 
the others are static and can be substituted out.  
The system looks like 
\begin{eqnarray*}
    \alpha E_t y_{t+1} + E_t p_{t+1} &=& \tau_g (y_t - e_a^\top x_t) 
                + \tau_p p_t + e_m^\top x_t \\
    p_t &=& \psi_g (y_t -e_a^\top x_t) + \psi_p E_t p_{t+1} + e_p^\top x_t 
\end{eqnarray*}
or 
\begin{eqnarray*}
    \left[
        \begin{array}{cc} 
            \alpha  &  1  \\  0 & \psi_p  
        \end{array}
    \right]
    \left[
        \begin{array}{c} 
            E_t y_{t+1} \\ E_t p_{t+1}  
        \end{array}
    \right]
    &=& 
    \left[
        \begin{array}{cc} 
            \tau_g & \tau_p  \\  -\psi_g & 1   
        \end{array}
    \right]
    \left[
        \begin{array}{c} 
            y_{t} \\ p_{t}  
        \end{array}
    \right]
    + 
    \left[
        \begin{array}{c} 
           e_m^\top - \tau_g e_a^\top \\
           \psi_g e_a^\top - e_p^\top   
        \end{array}
    \right]
    \left[  x_t \right] .
\end{eqnarray*}


\pagebreak
\subsection*{Appendix:  Hansen-Sargent formulas}

{\it Univariate version.\/}  
Here's a useful result from Hansen and Sargent (JEDC, 1980, p 14) 
and Sargent ({\it Macroeconomic Theory, 2e\/}, 1987, pp 303-304).  
An expectational difference equation with stationary forcing 
variable $x$ generates a 
``geometric distributed lead'':    
\begin{eqnarray*}
    y_t &=&    \lambda E_t y_{t+1} + x_t  \\
        &=&  \lambda E_t (\lambda  E_{t+1} y_{t+2} + x_{t+1}) +  x_t \\
        &=&  \sum_{j=0}^\infty \lambda^j  E_t x_{t+j} .
\end{eqnarray*}
If $ x_t = \sum_{j=0}^\infty \chi_j w_{t-j} = \chi(L) w_t $,  
with $w$ white noise, then what is $y_t$?  
A unique stationary solution $ y_t = \psi(L) w_t $
exists if $x$ is stationary and $ |\lambda|<1 $, 
but what is $\psi(L)$? 


Note how the distributed lead works.  
Conditional expectations of $x$ have the form 
\begin{eqnarray*}
    E_t x_{t+j}  &=&  [\chi(L)/L^j]_+ w_t 
                 \;=\;  \sum_{i=0}^\infty \chi_{j+i} w_{t-i} 
\end{eqnarray*}
(The subscript ``+'' means ignore negative powers of $L$.) 
Therefore the coefficient of $ w_{t-i}$ in the distributed lead is 
\begin{eqnarray*}
    \psi_i &=& \sum_{j=0}^\infty \lambda^j \chi_{i+j} .
\end{eqnarray*}
This tells us, for example, that if $x$ is MA($q$), then so is $y$:
if $ \chi_{j} = 0$ for $ j > q $, 
then $ \psi_j = 0 $ above the same limit.    

There's a ``lag notation'' version that expresses the result in 
compact form.
We're not sure whether it's all that useful for our purposes, 
but here it is.  
We're looking for a solution $y_t = \psi(L) w_t$ satisfying the 
expectational difference equation:  
\begin{eqnarray*}
    \psi(L) w_t &=& [\psi(L)/L]_+ w_t + \chi(L) w_t  .
\end{eqnarray*}
...
[flesh this out]  


See also Hansen and Sargent (``A note on Wiener-Kolmogorov prediction,''
ms, 1981).  


{\it Vector version.\/}
Here's a related result adapted from  
Ljungqvist and Sargent ({\it Recursive Macroeconomic Theory, 2e\/}, 2005, section 2.4).  
It extends the previous result to higher dimensional forcing processes
that can be expressed as stationary vector autoregressions.  
Consider the system
\begin{eqnarray*}
    y_t &=&    \lambda E_t y_{t+1} + u^\top x_{t}  \\
    x_{t+1}  &=&  A x_t + B w_{t+1} , 
\end{eqnarray*}
where $u$ is an arbitrary vector and $w$ is iid 
with mean zero and variance $I$.  
The solution in this case is 
\begin{eqnarray*}
    y_t &=&  \sum_{j=0}^\infty \lambda^j  u^\top E_t x_{t+j}  
        \;=\;  u^\top \sum_{j=0}^\infty \lambda^j  A^j x_{t} 
        \;=\;  u^\top (I-\lambda A)^{-1} x_t.
\end{eqnarray*}
[The last step follows from the matrix geometric series.]


There's a method of undetermined coefficients version of this.
Guess $ y_t = a^\top x_t$ for some vector $a$
(we know the solution has this form from what we just did).
Then the difference equation tells us
\begin{eqnarray*}
    a^\top x_t &=&  a^\top \lambda A x_t + u^\top x_t .
\end{eqnarray*}
Collecting terms in $x_t$ gives us 
$ a^\top =  u^\top (I-\lambda A)^{-1} $, as stated.
What's missing from this approach 
is an indication that $\lambda A$ must be stable.  


Here's a vector version.
Let $y$ be a vector with 
\begin{eqnarray*}
    y_t &=&  L E_t y_{t+1} + G x_{t} .      
\end{eqnarray*}
Use the usual law of motion for $x_t$.  
If we guess $y_t = F x_t$, substitution seems to give us 
\begin{eqnarray*}
    F &=& L F A + G .   
\end{eqnarray*}
How do we solve this for $F$?  
Is there a formula or are we stuck with numerical methods?  


\end{document}

Paper ideas
Interest rates and mon policy

1. basic features of interest rates
* where does persistence come from?  
* variation in short and long rates

2. changes in inflation and yield curve properties 
* Check Cogley-Primiceri-Sargent and Atkeson-Kehoe

Gregor's suggestions:  
Leeper discussion of Cochrane 
Phaneuf paper with Normandin in JME (04?)  


Related papers
* Canzoneri, Cumby, and Diba, JME, 07?:  errors in Euler eqs related to mon pol
* 

---------- Forwarded message ----------
From: David Backus <david.backus@gmail.com>
Date: Sep 16, 2006 10:41 AM
Subject: Re: Taylor rules
To: John Cochrane <john.cochrane@chicagogsb.edu>

John,

I'm probably trying your patience, but I've been thinking about this
because I'm curious how these NK models work and because Sargent keeps
giving me old papers on linear rational expectations models, which are
strangely addictive.  I think I've gone one step further than our last
exchange in Case 3 below.  The summary:

Model:  Take your eq (1,2), set r=0, and have shock x1 in eq (1), x2
in eq (2), with (x1,x2) independent.

Information assumptions:  Agents observe everything, but
econometricians (us) observe only i and pi. Plus phi > 1.

Case 1:  x2=0.  You can't infer x2 or estimate phi.  Using your first
two equations:  You observe i and pi, and let's throw out r to keep
things simple.  From i you can figure out (phi pi + x).  But that's
just E pi, which we already knew from our knowledge of pi's dynamics
-- and from i directly in eq (1).  So it seems like we can't infer x
with this structure.

Case 2:  x1=0.  Now you can infer x2 from i and pi [use eq (1)] and
therefore estimate phi.  Sounds a little like supply and demand:  you
need a shock to the other equation to identify the Taylor rule
parameter.

Case 3:  both shocks present.  Here it gets interesting.  In general,
i and pi will depend on both shocks, but it's not obvious how.  But if
you put structure on their dynamics, you can sometimes tell them
apart.  An example that seems to be in the CGG spirit (I can't say I
understand their logic completely, but they talk about the error in
the Taylor rule):  x1 ~ AR(1), x2 ~ white noise.  Then i ~ AR(1) and
pi ~ ARMA(1,1), which you can use to back out x1 and x2, hence
identify phi.  This case is moderately interesting, at least from a
technical point of view.  There's a lot of stuff like it in old HS
papers, which I'd guess will allow you to generalize the example.

You can argue separately whether you find the identification schemes
in Cases 2 and 3 persuasive, but I think they summarize the logic.

Dave 