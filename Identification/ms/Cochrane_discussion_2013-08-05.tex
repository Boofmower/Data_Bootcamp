\documentclass[12pt]{article}

\oddsidemargin=0.25truein \evensidemargin=0.25truein
\topmargin=-0.5truein \textwidth=6.0truein \textheight=8.75truein

%\RequirePackage{graphicx}
\usepackage{hyperref}
\usepackage{comment}
\usepackage{enumerate}
\usepackage[small,compact]{titlesec}

%\titleformat{\section}{\large\bfseries}{\thesection}{1em}{}
%\titleformat{\subsection}{\bfseries}{\thesection}{1em}{}

% document starts here
\begin{document}
\parskip=\bigskipamount
\parindent=0.0in
\thispagestyle{empty}
\today

John,

Thanks for the comments, they were useful in lots of ways.
We --- I'm including Mike and Stan here ---
hope this is part of an ongoing conversation about related issues,
so we thought we'd respond in some detail.
We've also rewritten the paper to address some of your comments,
and will do more once we've given this some thought.

%{\it The paper.\/}
First, let us say that we see the paper contributing to our understanding of the conditions required for identification.  It's not about whether these conditions are plausible, or whether the models themselves make sense.  We have some interest in both of these other issues,
but we thought it best to stick to the first one and give the paper a sharp focus.

We think we learned a few things along the way.
We found, for example, that some of our early guesses about how this worked
were wrong.
We thought we needed a shock in the other equation.
The ``Gertler example'' suggests something of the sort.
It turns out to have nothing to do with that.
What we need is a restriction on the Taylor rule shock.
A zero somewhere would do, or some kind of orthogonality condition.
If we go with orthogonality, then it has to be orthogonal to something,
and the other shock will work.
But any restriction will do.
We also thought knowing the yield curve would help.
It doesn't, at least with this problem.

More constructively, we found that we need to know something about the shock.
Period.
We need to observe it, or know enough about it to impose a restriction.
This applies to any forward-looking equation with an unobserved shock,
including the NK Phillips curve and its cost-push shock (Section 4).


{\it 1. This is a personal question not a citation question,
but I'm still puzzled whether the solutions you advocate are in the class I discuss or not.
%...  I haven't seen a
%single example of a successful restriction on the shock in this paper that I could use to actually
%identify the Taylor rule parameter.
\/}

To be honest, we're not sure, you have lots of stuff there.
%But purely as a matter of logic, we've shown you need a restriction --- one.
%Isn't that how the theory of identification works?

In the discussion that follows this comment, it seems to us that
you're combining two issues:  the theory of identification and the plausibility of restrictions.
We think it's useful to separate the issues because
(i)~they're logically distinct and (ii)~reasonable people can and will disagree about what is plausible and what is not.  It sounds presumptuous,
but to use an analogy,
we're doing the work of Koopmans, not Liu or Sims,
although we find both interesting.

On $s_{1t}$ specifically:  In 3.1 it's tied to consumption growth, hence plausibly observable in this model, although that's not our point.  More relevant to us:  the identification problem has the same form whether this shock is there or not.
See the last para of 3.1.
In 3.2, it's hard to say what kind of restrictions make sense,
which we think is one of the things Ken Singleton has been saying:
that is, what conceivable restrictions would you use in this model?
We've wondered the same thing, and come up empty.

Note, too, the large number of people who have estimated affine models and interpreted the short rate equation as a Taylor rule.  We haven't rubbed their faces in it, but they're plainly
not solving the identification problem. %, they're ignoring it.

%\pagebreak
{\it 2. If you can identify the Taylor rule, run just one regression!\/}

It's not a regression in these models, but we have some numerical examples in
Appendix B that work through the calculations you need to estimate the Taylor rule.

You mention data here.
We know it sounds wimpy, but identification is about whether you can estimate the parameters
of a given model, not whether the model is a good one.
Maybe that's what you mean by ``heroic assumptions,''
in which case the examples should do the trick.
%You say something similar in your paper (start of Section IV):
%``Determinacy and identification are properties of specific models, not
%general properties of variables and parameters.''

{\it 3. To illustrate the point, consider a simple Taylor rule...\/}

Hmmm...  If we understand what you're saying, yes, it's true, we assume agents know the model
and see the shocks.  So you ask:  how do they solve the identification problem?  To be glib, we just assume they do.  % on the theory that we’re addressing one question at at time.
That's the classical approach, and the one followed in the models we're discussing.
But we're puzzled by what else we might do.
If the agents don't know the model, what do they do?
What do they think others are doing?
If we're not careful, we get into one of those infinite regress problems:
who knows what, what do others think they know, etc.
That's an issue for any forward-looking model, not only NK models, and it's not an easy one to deal with.

Do you have something specific in mind?  If so, we'd love to hear it.

{\it 4. See my long discussion of Giannoni, Rotemberg and Woodford.\/}

You say this to bring up a different issue, but let us remind you of something you said
at the EFG meeting in New York when you gave your paper.
This is from memory, so the words won't be as clean as yours,
but our recollection is that you said something like:
NKs like to say their models are ``new'' and ``forward-looking,''
but when they talk informally it's all Old K language.

Yes, exactly!
This shows up most clearly, we think, when they switch back and forth between models
and VARs.
In the models, things are forward-looking and things react immediately to shocks.
In VARs, it's common (standard?) to say that shocks have no current impact on things.
In the piece you cite, Giannoni and Woodford are working in the VAR tradition.
You quote them on page 600 of your paper:
``we assume . . . that a monetary policy shock at date t has no effect on inflation, output or the real wage in that period.''
That sounds a lot like Sims to us.
And yes, that  pretty much does in the NK theory, where the rule can't be estimated
precisely because it affects inflation immediately.
What you say here sounds exactly right to us:
``This is an especially surprising result of a new-Keynesian model because
[these variables] are endogenous ... [and] jump in the new-Keynesian equilibrium.''

It's easy, we know, to take different approaches in different papers,
so maybe it's unfair to compare statements across papers.
We were hoping to contribute by providing a clean framework
for talking about restrictions
so that comparisons are more easily made.
Our guess of what's happening is that Sims has some affinity for the Old K thinking,
which he argue mimics lots of the stylized facts reasonably well.
So that shows up in lots of VAR work.
But it's an imperfect fit with NK models, as you note,
and mixing the two is likely to generate more confusion than insight.

%\pagebreak
{\it 5. New Keynesian macro.\/}

You don't say this, at least in these words,
but there's a question lurking behind that scenes that we sympathize with:
namely, is all this NK stuff a waste of our time?
Well, we've voted with our own actions and avoided it for most of our careers.
But there's value in meeting people halfway,
which is what we're trying to do here.
We've noticed you doing the same in your blog, and Steve Williamson as well.

We also find, as you do, that a lot of the financial data we use is nominal,
so we need to address money and inflation somehow.
Some of the old cash-in-advance models were one way of doing this.
A Taylor rule is another, and it has the advantage of being simple.
Thinking along these lines led us to early versions of your 2011 paper,
and we're still (slowly) working our way up the learning curve.
Along the way, we thought we might clarify issues as they come up.
%Identification is one.  Determinacy is another.
%The tendency to put shocks all over the place to make things ``fit'' is yet another.


In the meantime, let us know if you're ever in New York (Dave and Stan)
or Los Angeles (Mike).
We should continue this over a cold beer.


Cheers, \\
Dave, Mike, and Stan


\end{document}

John's response:  

A few items

"First, let us say that we see the paper contributing to our understanding of the
conditions required for identification. It's not about whether these conditions are
plausible,"

I'd disagree a bit. It's all about plausible. There are always assumptions that give identification. I can say "I assume that the number of angels that can dance on the head of a pin is equal to the number of raindrops that fall in my coffee cup left outside tomorrow." Well, that identifies all right. But is it plausible?

Here is a great condition for identification: There is no monetary policy shock. I_t = phi pi_t + 0. With a stochastic singularity you can identify phi perfectly! Alas it's not plausible. Giannoni and woodford do identify. The real question is whether their restrictions are plausible. Does the Fed really have zero response to expected future inflation, and does the economy really have zero response to policy changes within the quarter? Hmmm.

 "What we need is a restriction on the Taylor rule shock."
Absolutely! Though restrictions  on the functional form of the taylor rule -- same response to in and off equilibrium, exclusion restrictions, etc. matter too. Though one could say those are restrictions on the shock, since anything not in the rule ends up in the shock.

" Note, too, the large number of people who have estimated affne models and inter-
preted the short rate equation as a Taylor rule. We haven't rubbed their faces in it,
but they're plainly not solving the identification problem."

Identification is a property of the whole model, not just the Taylor rule equation. As in my paper, where Taylor rule in an old Keynesian model is nicely identified.

You have a great point here, which is how I'd write the paper. The affine models with Taylor rules have the same problem as new Keynesian models with Taylor rules -- the taylor rule parameter is not identified. It occurs, I think, for slightly different reasons. The Taylor rule in NK models is tied to an attempt to eliminate multiple equilibria while throwing away the equation (fiscal balance) which actually does the trick. I think the affine models have a different structure. But the non-identification works the same way.

If I were you, I would throw out the Gertler business, and actually the whole new Keynesian model business, and just dive in to Taylor rules in affine models. Rub their faces in it!  That would make a great paepr.

Point 3. In real rational expectations models, agents can learn the parameters by running regressions.

5. Is all this NK stuff a waste of our time? If I have to choose yes and no I choose yes. The project of writing a macro model with explicit frictions remains a good one. But I think they blew the equilibrium concept. As a result, their "impulse response functions" are not "how does putting more money in the economy cause inflation" it is "how does a policy shock get the economy to jump from one equilibrium to another after we impose an arbitrary equilibrium selection rule" So, I think the ingredients remain to be reassembled into a model with a unique global equilibrium. I do that by adding back the government valuation equation. There are other ways to do it. But the results are unlike anything they have produced, so, yes, the whole enterprise so far is a waste of time.

Beer always welcome! Same offer goes in Chicago. Keep me up on the work, as I am always interested in what you're up to

Best
John
