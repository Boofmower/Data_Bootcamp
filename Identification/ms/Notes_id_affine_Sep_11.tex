\documentclass[11pt]{article}

\oddsidemargin=0.25truein \evensidemargin=0.25truein
\topmargin=-0.5truein \textwidth=6.0truein \textheight=8.75truein

%\RequirePackage{graphicx}
\RequirePackage[hypertex]{hyperref}
\RequirePackage{comment}

\renewcommand{\thefootnote}{\fnsymbol{footnote}}
\newcommand{\var}{\mbox{\it Var\/}}
\def\stackunder#1#2{\mathrel{\mathop{#2}\limits_{#1}}}

% document starts here
\begin{document}
\parskip=\bigskipamount
\parindent=0.0in
\thispagestyle{empty}
\begin{flushright} Backus, Chernov, \& Zin \end{flushright}

\bigskip
\centerline{\Large \bf Notes on Identification of Taylor Rules%
\footnote{Working notes, no guarantee of accuracy or sense.}}
\centerline{(Started: July 23, 2010; Revised: \today)}

%State space.

\subsubsection*{Using yields}

Suppose under $Q,$ the state follows:
\begin{eqnarray}
     x_{t+1}=\Phi^Q x_t + \Sigma \epsilon_{t+1},
\end{eqnarray}
where $\Phi^Q$ is a diagonal matrix and the diagonal of the covariance matrix $\Sigma \Sigma^{\prime}$ is full of ones. These are restrictions that ensure identification of the $Q$ parameters based on yields. Then under $P,$ we have
\begin{eqnarray}
     x_{t+1}=\mu+\Phi x_t + \Sigma \epsilon_{t+1}.
\end{eqnarray}
There are no restrictions on $\mu$ and $\Phi.$ These two equations define risk premia $\Lambda_t=\Lambda_0+\Lambda_x^{\prime} x_t$ via:
\begin{eqnarray}
\label{L1}
   \Phi^Q&=& \Phi-\Sigma\Lambda_x^{\prime} \\
\label{L0}
   0 &=& \mu-\Sigma\Lambda_0.
\end{eqnarray}

Normally, one specifies the nominal interest rate as a linear function of these state variables, $i_t=\delta^{'}x_t.$ Yields data allow one to estimate the following set of parameters $(\Phi^Q, \Sigma, \mu, \Phi, \delta).$

Assume that inflation is a linear function of these state variables, $\pi_t=\alpha^{\prime} x_t.$ Because we imposed a restriction on the diagonal of the covariance matrix, $\alpha$ can be free.  Denote the real interest rate by $r_t.$ Then the nominal and real log pricing kernels are related via:
\begin{eqnarray}
    m_{t+1}^{\$}=m_{t+1} - \pi_{t+1} = - r_t - 1/2 \cdot \Lambda_t^{\prime} \Lambda_t - \Lambda_t^{\prime} \epsilon_{t+1} - \alpha^{\prime} x_{t+1}.
\end{eqnarray}
Therefore, the nominal interest rate rate is:
\begin{eqnarray}
    i_t&=&y^{\$}_t(1)=-\log P^{\$}_t(1) = -E_t(m_{t+1}-\pi_{t+1})-\frac{1}{2}\mbox{Var}_t(m_{t+1}-\pi_{t+1})\nonumber \\
    &=&r_t+\stackunder{E_t{\pi_{t+1}}}{\alpha^{\prime}\left(\mu+\Phi x_t \right)} -\stackunder{\mbox{convexity}}{1/2\cdot  \alpha^{\prime} \Sigma \Sigma^{\prime} \alpha} - \stackunder{\mbox{inflation premium}}{\alpha^{\prime} \Sigma \Lambda_t}.
\end{eqnarray}
This is a modern version of the Fisher equation that incorporates risk premia. Equations (\ref{L1}), (\ref{L0}) imply
\begin{eqnarray}
    i_t=r_t+\alpha^{\prime}\Phi^Q x_t  -1/2\cdot  \alpha^{\prime} \Sigma \Sigma^{\prime} \alpha.
\end{eqnarray}

This expression for the nominal interest rate must imply the same as the original $i_t=\delta^{'}x_t.$ If one specifies the real rate $r_t$ as a linear function of $x$, then one can explicitly compute $\delta$ if the set of loadings for $r_t$ and $\alpha$ are known, or vice-versa, one can infer $\alpha$ if $\delta$ and the set of loadings for $r_t$ are known. Here we do not want to take a stand on the real rate and treat $\delta$ as a nuisance parameter. Instead, we infer $\alpha$ from the monetary policy. We postulate monetary policy (MP) via the Taylor rule:
\begin{eqnarray}
      TR_t = \gamma_0 + r_t + \gamma_{\pi} \pi_t + \gamma^{\prime} x_t.
\end{eqnarray}

In equilibrium,  $i_t=TR_t,$ and we can solve for $\pi_t$ by solving for $\alpha:$
\begin{eqnarray}
    \alpha^{\prime}\Phi^Q x_t  -1/2\cdot  \alpha^{\prime} \Sigma \Sigma^{\prime} \alpha=  \gamma_0 + \gamma_{\pi}\alpha^{\prime}x_t+ \gamma^{\prime} x_t.
\end{eqnarray}
    We obtain a system of two equations:
 \begin{eqnarray}
     \alpha^{\prime}\Phi^Q x_t & = &\gamma_{\pi}\alpha^{\prime}x_t+ \gamma^{\prime} x_t \\
     -1/2\cdot  \alpha^{\prime} \Sigma \Sigma^{\prime} \alpha &=& \gamma_0.
 \end{eqnarray}
 The first equation implies:
 \begin{eqnarray}
     \alpha^{\prime}=\gamma^{\prime} \left(\Phi^Q-\gamma_{\pi} I\right)^{-1}.
 \end{eqnarray}
Because $\Phi^Q$ is diagonal, $\alpha_i=\gamma_i /(\phi_{ii}^Q-\gamma_{\pi}).$ %Thus, as we change MP responses, $\alpha$ changes as well. Thus, MP is not identified.

Compare this to the usual implementation of Taylor rules in affine term structure models. Authors specify the nominal interest rate in the form $i_t=\psi^{\prime}x_t+\psi_{\pi}\pi_t$ and interpret $\psi_{\pi}$ at the degree of the policy response. However, $\pi_t=\alpha^{\prime}x_t$ implies that $x_{jt}=\alpha_j^{-1}\pi_t-\alpha_{-j}^{\prime}\alpha_j^{-1}x_{-jt}.$ Therefore, $i_t=(\psi_{\pi}+\psi_j \alpha_j^{-1})\pi_t + f(x_{-jt})$ and the policy response is not identified.

Back to identification, the vector $y_t=(\pi_t, i_t)^{\prime}$ is related to the shocks $x_t$ (ignoring constants and $r_t$) via
\begin{eqnarray}
    y_t=Cx_t,
\end{eqnarray}
where, assuming  that $x=(x_1, x_2, \ldots, x_n)^{\prime},$
\begin{eqnarray}
    C=
\left(
\begin{array}{cccc}
\gamma_1 /(\phi_{11}^Q-\gamma_{\pi})  &   \gamma_2/(\phi_{22}^Q-\gamma_{\pi}) &  \ldots & \gamma_n/(\phi_{nn}^Q-\gamma_{\pi})   \\
\gamma_1 \phi_{11}^Q/(\phi_{11}^Q-\gamma_{\pi})  &   \gamma_2\phi_{22}^Q/(\phi_{22}^Q-\gamma_{\pi}) & \ldots &  \gamma_n\phi_{nn}^Q/(\phi_{nn}^Q-\gamma_{\pi} )
\end{array}
\right).
\end{eqnarray}
The $y_t$ follows the following VAR :
\begin{eqnarray}
   y_t= C \Phi C^{-1} y_{t-1} + \mbox{shock}.
\end{eqnarray}

%Suppose, we know $\Phi^Q$ from yields. Then we can estimate four parameters of $\Phi^*=C \Phi C^{-1},$ while we have 7 unknowns (4 elements of $\Phi$ and  the three $\gamma$'s). Suppose we care about $\gamma_{\pi}$ out of the three  $\gamma$'s. So fix $\gamma_i$ at some values, e.g., 1. We can try to achieve identification, by imposing restrictions on the risk premia $\Lambda_t.$
%For example, assume that $\lambda_{11}^1=0.$ Then, $\phi_{11}=\phi_{11}^Q.$

We can estimate $n^2$ parameters of $\Phi^*=C \Phi C^{-1},$ while we have $n+1$ unknowns ($n$ $\gamma$'s and $\gamma_{\pi}$, $\Phi$ and $\Phi^Q$ are known from yields). So if we have more than one factor in a model, all the parameters, including $\gamma_{\pi},$ should be identified.  Suppose $\Phi$ is not known either (why would this be? -- the model is estimated by NLS). Then $n+1$ parameters are not identified.  How could we achieve identification with additional restrictions?
\begin{enumerate}
     \item One strategy is to say that $\Lambda_t$ is too flexible and restrict risk premia, make $\Phi$ as close to $\Phi^Q$ as possible. This idea is connected to the literature stemming from the CP factor and constrained Sharpe ratios. For example, we can set $n+1$ elements of $\Lambda_1$ to zero. Note that Cochrane's example corresponds to $n=2$ with all elements $\Lambda_1$ equal to zero. This is an overly restricted model.
     \item Another strategy is to control how many $x$'s act as shocks to the Taylor rule. Note that at most, we can impose $n$ restrictions as there $n$ $\gamma$'s. Thus, this strategy on its own is not sufficient.
     \end{enumerate}

\subsubsection*{Using survey-based inflation expectations}

Another way to write the equilibrium $i=TR$ is
\begin{eqnarray}
   E_t(\pi_{t+1}) - \alpha^{\prime} \Sigma \Lambda_t = \gamma_{\pi} \pi_t + \gamma^{\prime} x_t.
\end{eqnarray}


\subsection*{Forward-looking Taylor rule}

[xxx Incomplete xxx]

We postulate monetary policy (MP) via the forward-looking Taylor rule:
\begin{eqnarray}
      TR_t = \gamma_0 + r_t + \gamma_{\pi}\frac{1}{\tau} E_t \left(\sum_{j=1}^{\tau} \pi_{t+j} \right) + \gamma^{\prime} x_t.
\end{eqnarray}
The expected inflation can be computed
\begin{eqnarray}
   E_t \left(\sum_{j=1}^{\tau} \pi_{t+j} \right)= \alpha^{\prime}\left(\Psi^{\tau}\mu+\Phi^{\tau}x_t\right),
\end{eqnarray}
where
\begin{eqnarray}
   \label{psinot}
   \Psi^{\tau}\equiv \sum_{k=0}^{\tau-1}\Phi^k=\left(I-\Phi\right)^{-1}\left(I-\Phi^{\tau}\right).
\end{eqnarray}
In equilibrium,  $i_t=TR_t,$ and we can solve for $\pi_t$ by solving for $\alpha:$
\begin{eqnarray}
    \alpha^{\prime}\Phi^Q x_t  -1/2\cdot  \alpha^{\prime} \Sigma \Sigma^{\prime} \alpha=  \gamma_0 + \tilde{\gamma}_{\pi}\alpha^{\prime}\left(\Psi^{\tau}\mu+\Phi^{\tau}x_t\right)+ \gamma^{\prime} x_t,
\end{eqnarray}
where $\tilde{\gamma}_{\pi}=\gamma_{\pi}/\tau.$ We obtain a system of two equations:
 \begin{eqnarray}
     \alpha^{\prime}\Phi^Q x_t & = &\gamma_{\pi}\alpha^{\prime}\Phi^{\tau}x_t+ \gamma^{\prime} x_t \\
     -1/2\cdot  \alpha^{\prime} \Sigma \Sigma^{\prime} \alpha &=& \gamma_0 + \tilde{\gamma}_{\pi}\alpha^{\prime}\Psi^{\tau}\mu.
 \end{eqnarray}
 The first equation implies:
 \begin{eqnarray}
     \alpha^{\prime}=\gamma^{\prime} \left(\Phi^Q-\tilde{\gamma}_{\pi} \Phi^{\tau}\right)^{-1}.
 \end{eqnarray}

 \subsection*{Using nominal yields, start with the real ones}

 Suppose under $Q,$ the state follows:
\begin{eqnarray}
     x_{t+1}=\Phi^Q x_t + \Sigma \epsilon_{t+1},
\end{eqnarray}
where $\Phi^Q$ is a diagonal matrix and the diagonal of the covariance matrix $\Sigma \Sigma^{\prime}$ is full of ones. These are restrictions that ensure identification of the $Q$ parameters based on yields.

Assume that the the real interest rate is
\begin{eqnarray}
     r_t=\rho_0 +  \rho_x^{\prime} x_t
\end{eqnarray}
and the real pricing kernel is
\begin{eqnarray}
     \log m_{t+1} = -r_t - \Lambda_t^{\prime}\Lambda_t - \Lambda_t  \epsilon_{t+1},
\end{eqnarray}
where
\begin{eqnarray}
    \Lambda_t = \Lambda_0+\Lambda_x^{\prime} x_t.
\end{eqnarray}

Now, suppose we have a nominal shock $z_t$ that is independent from $x.$ Under $Q$ $z$ is:
\begin{eqnarray}
    z_{t+1}=\phi_z^Q z_t+u_{t+1}.
\end{eqnarray}
Assume inflation is a linear function of all the factors:
\begin{eqnarray}
      \pi_t=\alpha_0+\alpha_x^{\prime}x_t+\alpha_z z_t.
\end{eqnarray}

Then the nominal and real log pricing kernels are related via:
\begin{eqnarray}
    \log m_{t+1}^{\$}=\log m_{t+1} - \pi_{t+1} = - r_t - 1/2 \cdot \Lambda_t^{\prime} \Lambda_t - \Lambda_t^{\prime} \epsilon_{t+1} - \alpha_0 - \alpha_x^{\prime} x_{t+1} - \alpha_z z_{t+1}.
\end{eqnarray}
Therefore, the nominal interest rate rate is:
\begin{eqnarray}
    i_t&=&y^{\$}_t(1)=-\log P^{\$}_t(1) = -E_t(\log m_{t+1}-\pi_{t+1})-\frac{1}{2}\mbox{Var}_t(\log m_{t+1}-\pi_{t+1})\nonumber \\
    &=&r_t+E_t{\pi_{t+1}} -\stackunder{\mbox{convexity}}{ [ \alpha_x^{\prime} \Sigma \Sigma^{\prime} \alpha_x+\alpha_z^2]/2} - \stackunder{\mbox{inflation premium}}{\alpha_x^{\prime} \Sigma \Lambda_t} \nonumber \\
    & = & \rho_0 +  \rho_x^{\prime} x_t + \alpha_0+\alpha_x^{\prime}\Phi^Q x_t + \alpha_z \phi_z^Q  z_t- [ \alpha_x^{\prime} \Sigma \Sigma^{\prime} \alpha_x+\alpha_z^2]/2.
\end{eqnarray}


We postulate monetary policy (MP) via the Taylor rule:
\begin{eqnarray}
      TR_t = \gamma_0 + \gamma_{\pi} \pi_t +\gamma_z z_t.
 \end{eqnarray}

In equilibrium,  $i_t=TR_t,$ and we can solve for $\pi_t$ by solving for $\alpha:$
\begin{eqnarray}
     \rho_0 +  \rho_x^{\prime} x_t + \alpha_0+\alpha_x^{\prime}\Phi^Q x_t + \alpha_z \phi_z^Q z_t- [ \alpha_x^{\prime} \Sigma \Sigma^{\prime} \alpha_x+\alpha_z^2]/2 = \gamma_0 + \gamma_{\pi} \pi_t+\gamma_z z_t.
\end{eqnarray}
    We obtain a system of two equations:
 \begin{eqnarray}
     \rho_x^{\prime} x_t + \alpha_x^{\prime}\Phi^Q x_t + \alpha_z \phi_z^Q z_t & = &\gamma_{\pi}\alpha_x^{\prime}x_t+ \gamma_{\pi} \alpha_z z_t +\gamma_z z_t\\
    \rho_0 + \alpha_0 -[ \alpha_x^{\prime} \Sigma \Sigma^{\prime} \alpha_x+\alpha_z^2]/2 &=& \gamma_0 +\gamma_{\pi}\alpha_0.
 \end{eqnarray}
 The first equation implies:
 \begin{eqnarray}
  & &    \rho_x^{\prime}+\alpha_x^{\prime}\Phi^Q=\gamma_{\pi}\alpha_x^{\prime} \\
  & &    \alpha_z \phi_z^Q = \gamma_{\pi} \alpha_z + \gamma_z
 \end{eqnarray}
 and
 \begin{eqnarray}
    \alpha_x&=&-\rho_x^{\prime}(\Phi^Q-\gamma_{\pi}I)^{-1} \\
    \alpha_z&=&\gamma_{z}/(\phi_z^Q-\gamma_{\pi})
 \end{eqnarray}

 Back to identification, the vector $y_t=(\pi_t, i_t)^{\prime}$ is related to the shocks $\tilde{x}_t=(x_t^{\prime},z_t)^{\prime}$ via (ignoring constants)
\begin{eqnarray}
    y_t=C \tilde{x}_t,
\end{eqnarray}
where, assuming  that $x=(x_1, x_2, \ldots, x_n)^{\prime},$
\begin{eqnarray}
    C=
\left(
\begin{array}{cccc}
-\rho_{x,1} /(\phi_{11}^Q-\gamma_{\pi})  &    \ldots &-\rho_{x,n} /(\phi_{nn}^Q-\gamma_{\pi})  &  \gamma_{z}/(\phi_z^Q-\gamma_{\pi})\\
\rho_{x,1} (1-\phi_{11}^Q/(\phi_{11}^Q-\gamma_{\pi}))   & \ldots &  \rho_{x,n} (1-\phi_{nn}^Q/(\phi_{nn}^Q-\gamma_{\pi}) ) &  \gamma_{z}\phi_z^Q/(\phi_z^Q-\gamma_{\pi})
\end{array}
\right).
\end{eqnarray}
The $y_t$ follows the following VAR :
\begin{eqnarray}
   y_t= C \Phi C^{-1} y_{t-1} + \mbox{shock}.
\end{eqnarray}
We can estimate $(n+1)^2$ parameters of $\Phi^*=C \Phi C^{-1},$ while we have $2$ unknowns ( $\gamma_{\pi}$ and $\gamma_z$, $\rho_x,$ $\Phi$ and $\Phi^Q$ are known from yields). So if we have more than one factor in a model, all the parameters, including $\gamma_{\pi},$ should be identified.

 \subsection*{Square-root case}




\end{document}

\begin{comment}

and the nominal pricing kernel is
\begin{eqnarray}
      \log m_{t+1}^{\$} = - i_t - \tilde{\Lambda}_t^{\prime}\tilde{\Lambda}_t -\tilde{ \Lambda}_t  ( \epsilon_{t+1}^{\prime}, u_{t+1} )^{\prime},
\end{eqnarray}
where
\begin{eqnarray}
     \tilde{\Lambda}_t =
\left(
\begin{array}{c}
\Lambda_t+\alpha_x     \\
   \alpha_z
\end{array}
\right).
\end{eqnarray}

Therefore,under $P,$ we have
\begin{eqnarray}
     x_{t+1}=\mu+\Phi x_t + \Sigma \epsilon_{t+1},
\end{eqnarray}
where
\begin{eqnarray}
\label{L1}
   \Phi&=& \Phi^Q+\Sigma\Lambda_x^{\prime} \\
\label{L0}
   \mu &=& \Sigma\Lambda_0.
\end{eqnarray}
Then
\begin{eqnarray}
     z_{t+1}=\mu_z+\phi_z z_t + u_{t+1},
\end{eqnarray}
where

 \end{comment}
