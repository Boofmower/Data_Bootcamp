\documentclass[11pt]{article}

\oddsidemargin=0.25truein \evensidemargin=0.25truein
\topmargin=-0.5truein \textwidth=6.0truein \textheight=8.75truein

%\RequirePackage{graphicx} 
\RequirePackage[hypertex]{hyperref}
\RequirePackage{comment}

\renewcommand{\thefootnote}{\fnsymbol{footnote}}
\newcommand{\var}{\mbox{\it Var\/}}
\def\stackunder#1#2{\mathrel{\mathop{#2}\limits_{#1}}}

% document starts here
\begin{document}
\parskip=\bigskipamount
\parindent=0.0in
\thispagestyle{empty}
\begin{flushright} Backus, Chernov, \& Zin \end{flushright}

\bigskip
\centerline{\Large \bf Notes on how to solve for tau given exclusion restrictions%
\footnote{Working notes, no guarantee of accuracy or sense.}}
\centerline{(Started: July 23, 2013; Revised: \today)}

%State space.

Consider the representative-agent case. The interest rate equation (9) implies:
\begin{eqnarray*}
   a^{\top}=\alpha d_1^{\top}A+b^{\top}A.
\end{eqnarray*}
The equilibrium condition implies
\begin{eqnarray*}
      a^{\top}=\tau b^{\top} +d_2^{\top}.
\end{eqnarray*} 
Now, consider the exclusion restriction requiring independence of the two shocks: $d_1^{\top} V_x d_2=0.$ Use the expressions above to substitute for $d_1$ and $d_2:$
\begin{eqnarray*}
    \alpha^{-1}(a^{\top}-b^{\top}A)A^{-1} V_x (a-\tau b)=0.
\end{eqnarray*}
Therefore,
\begin{eqnarray*}
    \tau = (a^{\top}-b^{\top}A)A^{-1} V_x a  [(a^{\top}-b^{\top}A)A^{-1} V_x b]^{-1}.
\end{eqnarray*}

Is this exclusion restriction rotation-invariant? Consider $x_t^{\prime}=Qx_t$ and the corresponding constraint  $d_1^{\prime \top} V_x^{\prime} d_2^{\prime}=0,$ where $d_i^{\prime \top} = d_i^{\top}Q^{-1}$ (see section 5.1).
Next,
\begin{eqnarray*}
    V_x^{\prime}=E(x_t^{\prime}x_t^{\prime\top})=E(Qx_tx_t^{\top}Q^{\top})=QV_xQ^{\top}. 
\end{eqnarray*}
Therefore, the exclusion restriction is:
\begin{eqnarray*}
    0=d_1^{\prime \top} V_x^{\prime} d_2^{\prime}=d_1^{\top}Q^{-1} QV_xQ^{\top} (Q^{-1})^{\top} d_2 = d_1^{\top} V_x d_2.
\end{eqnarray*}
QED

My guess is that this works because all elements of the exclusion restrictions are changing with rotation in a specific way dictated by this rotation. This allows rotation itself, as manifested by the matrix $Q$ to cancel out. Consider, a more general restriction $f^{\top}d_2=0$. The restriction above is a particular case if $f^{\top}=d_1^{\top} V_x.$ For $x^{\prime}_t$ the restriction becomes $f^{\prime \top}d_2^{\prime}=0.$ But what is $f^{\prime}$? $f^{\prime}=Qf,$ so we cannot git rid of $Q$, in general.


\end{document}
