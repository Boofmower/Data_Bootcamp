\documentclass[11pt]{article}

\oddsidemargin=0.25truein \evensidemargin=0.25truein
\topmargin=-0.5truein \textwidth=6.0truein \textheight=8.75truein

\RequirePackage{comment}
\RequirePackage{graphicx}
\RequirePackage{hyperref}
\usepackage{amssymb}
\usepackage{amsfonts}
\usepackage{booktabs}

\renewcommand{\thefootnote}{\fnsymbol{footnote}}
\newcommand{\N}{\mathcal{N}}

\usepackage[small, compact]{titlesec}

\begin{document}
\thispagestyle{empty}
\parindent = 0.0in
\parskip = \bigskipamount

\begin{flushright}
Backus, Chernov \& Zin
\end{flushright}

\begin{center}
{\Large\bf Notes on Identification in VARs\footnote{%
Working notes, no guarantee of accuracy or sense.}
} \\
February 15, 2013; last revision \today
\end{center}

Summary of work identifying monetary policy in VARs.
Prepared for ``Identifying Taylor rules.''
Numbering follows sections numbers in the originals. 


\section*{Watson, handbook chapter, 1994}





\section*{Leeper, Sims, and Zha, Brookings Papers, 1996}

stuff..


\section*{Sims and Zha, Macro Dynamics, 2006}




\section*{Christiano, ``Sims and VARs,'' 2012}


\section*{Cooley and Dwyer, J Econometrics, 1998}

Page 77:  ``Any such restriction implies a corresponding instrumental variable.'' 
They cite Hausman and Taylor, Econometrica, 1983, 
``Identification in linear simultaneous equations models with covariance restrictions: an instrumental variables interpretation.''





\end{document}
