\documentclass[12pt]{article}

\oddsidemargin=0.50truein
\evensidemargin=0.50truein
\textwidth=5.50truein
\topmargin=-0.250truein
\textheight=8.25truein

\usepackage{comment}
\usepackage{graphicx}
\usepackage{booktabs}
\usepackage{amsfonts}
\usepackage{hyperref}
%\usepackage{microtype}

% from Chad Jones
\urlstyle{rm}   % change font for url's.
\hypersetup{
    colorlinks=true,        % kills boxes
    allcolors=blue,
    pdftitle={Identifying Taylor Rules},
    pdfauthor={Backus, Chernov, and Zin},
    pdfstartview={FitV},
    pdfpagemode={UseNone},
    pdfnewwindow=true,      % links in new window
%    pdfpagelabels=true,
%    linkcolor=blue,         % color of internal links
%    citecolor=blue,         % color of links to bibliography
%    filecolor=blue,         % color of file links
%    urlcolor=blue           % color of external links
% see:  http://www.tug.org/applications/hyperref/manual.html
}

\usepackage[compact, small]{titlesec}
%\titleformat{\section}{\large\bfseries}{\thesection}{1em}{}
%\titleformat{\subsection}{\bfseries}{\thesection}{1em}{}


\renewcommand{\thefootnote}{\fnsymbol{footnote}}
\newcommand{\N}{\mathcal{N}}
\renewcommand{\epsilon}{\varepsilon}
\newcommand{\var}{\mbox{\it Var\/}}

% next three lines for biblio setup
\newcommand{\paperstart}[0]{\par\penalty 1 \hangindent=32pt \hangafter=1}
\newcommand{\paperend}[0]{\penalty -1 \par\vskip 0pt plus 0pt minus 0pt}
%\newcommand{\paperend}[0]{\penalty -1 \par\vskip 15pt plus 25pt minus 10pt}
\newcommand{\paper}[1]{\paperstart #1 \paperend}

% Example counter and macro
\newcounter{remark}
\setcounter{remark}{1}
\newcommand{\remark}[1]{%
    \noindent{\it Remark \arabic{remark} (#1).\/}%
    \addtocounter{remark}{1}
    }

% Table and figure counters and macros
\newcounter{tab}
\setcounter{tab}{0}
\newcommand{\tab}[2]{%
    \refstepcounter{tab}%
    \label{#2}%
    \noindent{\large\bf Table \arabic{tab} \\ #1}%
    }
\newcounter{fig}
\setcounter{fig}{0}
\newcommand{\fig}[2]{%
    \refstepcounter{fig}
    \label{#2}
    \noindent{\large\bf Figure \arabic{fig} \\ #1}%
    }

\begin{document}
\thispagestyle{empty}
\newlength{\oldparindent}
\oldparindent=\parindent
\parindent=0.0in

%\begin{comment}
\vspace*{1.0in}  %% 11pt
%  ** use LARGE with 11pt, Large with 12pt
\noindent{\Large\bf Identifying Taylor rules in macro-finance \vspace*{0.1in} \\ models\footnote{%
The project started with our reading of John Cochrane's paper on the same subject
and subsequent emails and conversations with him and Mark Gertler.
We thank them both.
We also thank Rich Clarida, Patrick Feve, Ken Singleton,
Gregor Smith, and Tao Zha for comments on earlier drafts,
and participants in seminars at, and conference sponsored by,
New York University, the Swedish House of Finance, and Toulouse.
}}

\vspace{0.5in}
\noindent{%
David Backus,\footnote{
        Stern School of Business, New York University, and NBER;
        david.backus@nyu.edu.}
Mikhail Chernov,\footnote{
        Anderson School of Management, UCLA, and CEPR; 
        mikhail.chernov@anderson.ucla.edu.}
and Stanley Zin\footnote{
        Stern School of Business, New York University, and NBER;
        stan.zin@nyu.edu.}
}

\vspace{0.4in}
{\today} %\\ Working draft
%\\  Please do not quote or distribute without permission


\vfill
{

\bigskip
%%\medskip
%\noindent{\bf JEL Classification Codes: }  E43, E52, G12.
%
%\medskip
%{\noindent{\bf Keywords: } forward-looking models; information sets; monetary policy;
%exponential-affine models.}
%\smallskip
}
%\end{comment} 

%\end{document}
% Start of text ***************************************************************
\newpage
\parskip=\bigskipamount
\setcounter{page}{1} \thispagestyle{empty}
\newlength{\oldbaselineskip}
\setlength{\oldbaselineskip}{\baselineskip}
\baselineskip = 1.2\baselineskip
\noindent{\bf Abstract}

\medskip
\noindent{%
Identification problems arise naturally in forward-looking models
when agents observe more than economists.
We illustrate the problem in several macro-finance models with Taylor rules.
Identification of the rule's parameters requires restrictions on the form of the policy shock.
We show how such restrictions work when we observe the state of the economy directly
or infer it from observable macro variables or asset prices.


%\end{document} 
\newpage
\parskip=\bigskipamount
\setcounter{page}{1} \thispagestyle{empty}
%\newlength{\oldbaselineskip}
\setlength{\oldbaselineskip}{\baselineskip}
\baselineskip = 1.2\baselineskip
\section{Introduction}

The field of macro-finance has the potential to give us deeper insights
into macroeconomics and macroeconomic policy by using information
from both aggregate quantities and asset prices.
The link between bond-pricing and monetary policy seems particularly promising
%in this respect
if central banks implement monetary policy through short-term interest rates,
as they do in models with Taylor rules.

If the combination of macroeconomics and finance holds promise, it also raises challenges.
We address one of them here: the challenge of identifying monetary policy
parameters in a modern macroeconomic model.
Identification problems arise in many economic models,
but they play a particularly important role in assessments
of monetary policy.
If we see that the short-term interest rate rises with inflation,
does that reflect the policy of the central bank or something else?
How would we know?
Since identification is a feature of models,
the question is what we need in a model to be able to identify the
parameters governing monetary policy.

We illustrate the problem and point to its possible resolution in several
examples that combine elements of New Keynesian and macro-finance models.
The source of the identification problem here is that we economists do not
observe the shock to monetary policy, although the agents populating the model do.
As a result, it's difficult, and perhaps impossible,
to disentangle the systematic aspects of monetary policy from shocks to it.
Without more information about the shock,
we may not be able to identify the parameter tying, say,
interest rate policy to the inflation rate.
%
This issue about the difference between what agents and economists observe
goes back (at least) to work by Hansen and Sargent (1980, 1991),
who considered differences in the information sets of agents
and economists in dynamic rational expectations models.
In our examples, the issue is the Taylor rule shock.
If agents observe it but economists do not,
then we need restrictions on the shock to identify
the Taylor rule's parameters.
The identification problem in these models is different
from the classical Cowles Commission work on simultaneous equation systems.
The central issue is whether we observe the shock, not whether
exogenous variables show up in the right configuration.
If we observe the shock, identification follows immediately.
If not, then we need restrictions on its coefficients.

We show how this works in a series of examples that illustrate
how information about shocks interacts with dynamics in forward-looking rational expectations
models.
The first example is adapted from Cochrane (2011)
and consists of the Fisher equation and a Taylor rule.
Later examples introduce exponential-affine pricing kernels
and New Keynesian Phillips curves.
We find it helpful to separate issues concerning the observation of shocks
from those concerning the observation of the underlying state of the economy.
If the state is observed, a clear conclusion emerges:
we need restrictions on the shock to the Taylor rule to
identify the rule's parameters.
In our examples, we need one restriction for each parameter to be estimated.
If the state is not observed, nothing changes:  The same kinds of restrictions
deliver identification.

If this seems clear to us now, it was not when we started.
We thought, at first, that identification required shocks in other equations.
We find, instead, that the conditions for identification don't change
when we eliminate other shocks. %, or even when we eliminate the other equations.
It depends entirely on what we know about the shock in the equation of interest,
the Taylor rule.
We also thought that knowledge of the term structure of interest rates
might help with identification.
We find, instead, that knowing the term structure can be helpful
in observing the state, but knowing the state is not enough to identify the Taylor rule.
What matters in all of these examples is whether we observe the shock to the Taylor rule,
and if we do not, whether we have restrictions on its form.

Identification depends, then,
on our ability to observe shocks or, failing that, to impose restrictions
on their coefficients.
The answer to the next question is less clear:
What kinds of restrictions are plausible?
Reasonable people can and will have different opinions about this.
We leave it for another time.


\section{The problem}
\label{sec:examples}

Two examples illustrate the nature of identification problems in
macro-finance models with Taylor rules.
The first comes from Cochrane (2011).
The second is an exponential-affine bond-pricing model.
The critical ingredient in each is what we observe.
We assume that economic agents observe everything,
but we economists do not.
In particular, we do not observe the shock to the Taylor rule.
The question is how this affects our ability to
infer the Taylor rule's parameters.
We provide answers for these two examples and
discuss some of the questions they raise about identification
in similar settings.


\subsection{Cochrane's example}
\label{sec:example-cochrane}

Cochrane's example consists of two equations,
an asset pricing relation (the Fisher equation)
and a Taylor rule (which depends only on inflation):
\begin{eqnarray}
    i_t &=& E_t \pi_{t+1} %+ s_{1t}
        \label{eq:cochrane-euler} \\
    i_t &=& \tau \pi_t + s_{t} .
        \label{eq:cochrane-taylor}
\end{eqnarray}
Here $i_t$ is the (one-period) nominal interest rate,
$\pi_t$ is the inflation rate,
and $s_t$ is a monetary policy shock.
The Taylor rule parameter $\tau>1$ describes how aggressively
the central bank responds to inflation.
This model is extremely simple,
but it's enough to illustrate the identification problem.

Let us say, to be specific, that
the shock is a linear function of a state vector $x_t$,
$s_t = d^\top x_t$,
and that $x_t$ is autoregressive,
\begin{eqnarray}
    x_{t+1} &=&  A  x_t + C w_{t+1} ,
    \label{eq:state-dynamics}
\end{eqnarray}
with $ A $ stable and disturbances $\{ w_t \} \sim \mbox{NID}(0,I)$.
Although simple, this structure is helpful
for clarifying the conditions that allow identification.
It also allows easy comparison to models ranging
from exponential-affine to vector autoregressions.
For later use, we denote the covariance of innovations by
$ V_w = C E(w w^\top)C^\top = C C^\top $
and the covariance matrix of the state by $V_x= E(x x^\top)$,
the solution to $V_x = AV_x A^{\top} +CC^{\top}$.


We solve the model by standard methods;
see Appendix \ref{app:hs-formulas}.
Here and elsewhere, we assume agents know the model
and observe all of its variables.
Equations (\ref{eq:cochrane-euler}) and (\ref{eq:cochrane-taylor})
imply the forward-looking difference equation
or rational expectations model
\begin{eqnarray*}
    E_t \pi_{t+1}&=& \tau \pi_t + s_t .
\end{eqnarray*}
The solution for inflation has the form $\pi_t = b^\top x_t $
for some coefficient vector $b$ to be determined.
Then $ E_t \pi_{t+1} = b^\top E_t x_{t+1} = b^\top  A  x_t$.
Lining up terms, we see that $b$ satisfies
\begin{eqnarray}
      b^\top  A   &=& \tau b^\top + d^\top
      \;\;\;\;\Rightarrow \;\;\;\; b^\top \;\;=\;\; - d^\top (\tau I -  A )^{-1}  .
    \label{eq:cochrane-solution}
\end{eqnarray}
This is the unique stationary solution if $ A $ is stable
(eigenvalues less than one in absolute value)
and $\tau> 1$ (the so-called Taylor principle).
Equation (\ref{eq:cochrane-euler}) then gives us
$ i_t = a^\top x_t$ with $a^\top = b^\top A $.


Now consider estimation. Do we have enough information
to estimate the Taylor rule parameter $\tau$?
We might try to estimate equation (\ref{eq:cochrane-taylor})
by running a regression of $i_t$ on $\pi_t$, with the shock $s_t$ as the residual.
That won't work because $s_t$ drives both $i_t$ and $\pi_t$,
and we need to distinguish its direct effect on $i_t$ from its indirect effect through $\pi_t$.
Least squares would deliver a coefficient of
$\mbox{Var}(\pi)^{-1} \mbox{Cov}(\pi,i) = (b^{\top}V_x b)^{-1} b^{\top} V_x a$,
which is not in general equal to $\tau$.

How then can we estimate $\tau$?
The critical issue is whether we observe the shock $s_t$.
Let us say that we --- the economists --- observe the state $x_t$,
but may or may not observe the shock $s_t$ or the coefficient vector $d$ that connects it to
the state.
Because we observe $x_t$, we can estimate $ A $ and $V_x$.
We can also estimate the parameter vectors $a$ and $b$
connecting the interest rate and inflation to the state.
If we observe the shock $s_t$, then we can estimate the parameter vector $d$.
We now have all the components of (\ref{eq:cochrane-solution}) but $\tau$,
which we can infer.
Evidently the Taylor rule parameter is identified.
In fact, it is over-identified.
If $x$ has dimension $n$, we have $n$ equations that each determine $\tau$.
% ?? how identified?

However, if we don't observe $s_t$, and therefore do not know $d$, we're in trouble.
In economic terms,
we can't distinguish the effects on the interest rate
of inflation (the parameter $\tau$) and the shock (the coefficient vector $d$).
In this model, there's not much we can do about that.


\subsection{An exponential-affine example}
\label{sec:example-affine}

Another perspective on the identification problem
is that we can't distinguish the pricing relation (\ref{eq:cochrane-euler})
from the Taylor rule (\ref{eq:cochrane-taylor}).
Sims and Zha (2006, page 57) put it this way:
``The ... problem ...  is that the
Fisher relation is always lurking in the background.
The Fisher relation connects current nominal rates to expected future inflation rates
and to real interest rates[.] ...
So one might easily find an equation that had the form of the ... Taylor rule,
satisfied the identifying restrictions, but was something other than a policy reaction function.''
Cochrane (2011, page 598) echoes the point:
``If we regress interest rates on output and inflation, how do we know
that we are recovering the Fed's policy response, and not the parameters
of the consumer's first-order condition?''
We'll see exactly this issue in the next example,
in which we introduce an exponential-affine model into the problem.


Consider, then, an exponential-affine model of interest rates,
a structure that's widely used in finance,
in which bond yields are linear functions of the state.
In the macro-finance branch of this literature,
the state includes macroeconomic variables like inflation and output growth.
Examples include
Ang and Piazzesi (2003),
Chernov and Mueller (2012),
Jardet, Monfort, and Pegoraro (2012),
Moench (2008),
Rudebusch and Wu (2008),
and Smith and Taylor (2009).
In these models the short rate depends on,
among other things, inflation.

An illustrative example follows from the log pricing kernel,
\begin{eqnarray}
    m^{\$}_{t+1} &=& - \lambda^\top \lambda/2 - \delta^\top x_{t} + \lambda^\top w_{t+1} ,
    \label{eq:example2-logm}
\end{eqnarray}
and the linear law of motion (\ref{eq:state-dynamics}).
Here the nominal (log) pricing kernel $m^{\$}_{t}$ is
connected to the real (log) pricing kernel $m_{t}$
by $m^{\$}_t = m_t  - {\pi_t} $.
The one-period nominal interest rate is then
\begin{eqnarray}
     i_t &=& - \log E_t \exp( m_{t+1}-\pi_{t+1})
        \label{eq:macrofin-euler} \\
        &=& - \log E_t \exp(m^{\$}_{t+1})
        \;\;=\;\; \delta^\top x_t .
        \label{eq:example2-shortrate}
\end{eqnarray}
If we observe the state $x_t$, we can estimate $\delta$
by projecting the interest rate on it.

If the first element of $x_t$ is the inflation rate,
it's tempting to interpret equation (\ref{eq:example2-shortrate}) as a Taylor rule,
with the first element of $\delta$ the inflation coefficient $\tau$.
But is it?
The logic of equation (\ref{eq:example2-shortrate})
is closer to the asset-pricing relation, equation (\ref{eq:cochrane-euler}),
than to the Taylor rule, equation (\ref{eq:cochrane-taylor}).
But without more structure, we can't say whether it's one, the other,
or something else altogether.
This is, of course, the point made by Sims and Zha
and echoed by Cochrane.
Joslin, Le, and Singleton (2013) make a similar point.

More formally, consider an interpretation
of (\ref{eq:example2-shortrate}) as a Taylor rule (\ref{eq:cochrane-taylor}).
Since we observe inflation $\pi_t$ and the state $x_t$,
we can estimate the coefficient vector $b$ connecting the two:  $\pi_t = b^\top x_t$.
Then the Taylor rule implies
\begin{eqnarray*}
    i_t &=& \tau \pi_t + s_t
        \;\;=\;\; \tau b^\top x_t + d^\top x_t .
\end{eqnarray*}
Equating our two interest rate relations gives us
$\delta^\top = \tau b^\top + d^\top$.
It's clear, now, that we have the same difficulty we had in the previous example:
If we do not know the shock parameter $d$,
we cannot infer $\tau$ from estimates of $\delta$.
If $x_t$ has dimension $n$, we have $n$ equations to solve for
$n+1$ unknowns ($d$ and $\tau$).

If we interpret (\ref{eq:example2-shortrate}) as an asset pricing relation,
then it's evident that we can't distinguish asset pricing (represented by $\delta$)
from monetary policy (represented by $\tau b + d $) without more information
about the shock coefficients $d$.
Generalizing the asset pricing relation from (\ref{eq:cochrane-taylor})
to (\ref{eq:example2-shortrate}) has no effect on this conclusion.


\subsection{Discussion}
\label{sec:examples-discussion}

These examples illustrate the challenge we face in identifying the parameters
of the Taylor rule,
but they also suggest follow-up questions that might lead to a solution.

One such question is whether we can put shocks in other places
and use them for identification.
Gertler (private communication) suggests putting a shock
in Cochrane's first equation, so that the example becomes
\begin{eqnarray*}
    i_t &=& E_t \pi_{t+1} + s_{1t} \\
    i_t &=& \tau \pi_t + s_{2t} .
\end{eqnarray*}
Can the additional shock identify the Taylor rule?

Suppose, as Gertler suggests, that $s_{1t}$ and $s_{2t}$ are independent.
Then if $s_{1t}$ is observed,
we can use it as an instrument for $\pi_t$ to estimate the Taylor rule
equation, which gives us an estimate of $\tau$.
Given $\tau$, we can then back out the shock $s_{2t}$.
We'll see in the next section that this example is misleading
in one respect ---  we do not need a shock in the other equation ---
but there are two conclusions here of more general interest.
One is that identification requires a restriction on the Taylor rule shock.
Here the restriction is independence, but in later examples
other restrictions serve the same purpose.
The other is that identifying $\tau$
and backing out the unobserved shock are complementary activities.
Generally if we can do one, we can do the other.

A second question is whether we can use long-term interest rates to
help with identification.
The answer is no if the idea is to use long rates to observe the state.
In exponential-affine models, the state spans bond yields of all maturities.
In many cases of interest, we can invert the mapping and express
the state as a linear function of a subset of yields.
In this sense, we can imagine using a vector of bond yields
to observe the state.
%
We have seen, though, that observing the state is not enough.
We observe the state in both examples,
yet cannot identify the Taylor rule.
We explore the issue of state observability further in Section \ref{sec:observability}.


\section{Macro-finance models with Taylor rules}
\label{sec:macrofin}

Macro-finance models, which combine elements of
macroeconomic and asset-pricing models,
bring evidence from both macroeconomic and financial variables to bear
on our understanding of monetary policy.
It's not easy to reconcile the two,
but if we do, we gain perspective that's
missing from either approach on its own.

We show how Gertler's insight can be developed to identify
the Taylor rule in such models.
We use two examples, one based on a representative agent,
the other on an exponential-affine model.
We explore identification in these models when we observe the state,
the short rate, and inflation,
but not the shock to the Taylor rule.
The identification issues are the same:  we need one restriction on the shock
to identify the (one) policy parameter.


\subsection{A representative-agent model}
\label{sec:rep-agent}

One line of macro-finance research combines representative-agent
asset pricing with a rule governing monetary policy.
Gallmeyer, Hollifield, and Zin (2005) is a prominent example.
We simplify their model,
using power utility instead of recursive preferences and
a simpler law of motion for the state.

The model consists of equation (\ref{eq:macrofin-euler}) plus
\begin{eqnarray}
    m_t &=& - \rho - \alpha g_t \label{eq:macrofin-real-m} \\
    g_t &=& g + s_{1t}   \label{eq:con-growth} \\
    i_t &=& r + \tau \pi_t + s_{2t} .
        \label{eq:macrofin-taylor}
\end{eqnarray}
Equations (\ref{eq:macrofin-euler}) and (\ref{eq:macrofin-taylor}) mirror
the two equations of Cochrane's example.
The former is
a more complex version of equation (\ref{eq:cochrane-euler})
that represents the finance component of the model.
The latter is a Taylor rule, representing monetary policy.
Equations (\ref{eq:macrofin-real-m}) and (\ref{eq:con-growth}) characterize the real pricing kernel.
The first is the logarithm of the marginal rate of substitution
of a power utility agent with discount rate $\rho$,
curvature parameter $\alpha$,
and log consumption growth $g_t$.
The second connects fluctuations in log consumption growth to a shock $s_{1t}$.
As in Section \ref{sec:examples}, the state $x_t$ follows the law of motion (\ref{eq:state-dynamics})
and shocks are linear functions of it: $s_{it} = d_i^\top x_t $ for $i=1,2$.
For simplicity, we choose $r$ to reconcile the two interest rate equations,
which makes mean inflation zero.

The solution now combines asset pricing with a forward-looking difference equation.
We posit a solution of the form $\pi_t = b^\top x_t$.
Solving (\ref{eq:macrofin-euler}) then gives us
\begin{eqnarray}
    i_t   &=&  \rho+\alpha g - V_m/2  + a^\top x_t ,
    \label{eq:macrofin-euler-solved}
\end{eqnarray}
with
\begin{eqnarray*}
    a^\top &=& (\alpha d_1^\top + b^\top)  A  \\
    V_m     &=& a^\top C C^\top a .
\end{eqnarray*}
Note that the short rate equation (\ref{eq:macrofin-euler-solved}) now has a shock,
as Gertler suggests.
Equating (\ref{eq:macrofin-taylor}) and (\ref{eq:macrofin-euler-solved}) gives us
\begin{eqnarray*}
    \left(\rho+\alpha g - V_m/2\right)
    + (\alpha d_1^\top + b^\top)  A  x_t
    &=&
    r + (\tau b^\top + d_2^\top ) x_t.
\end{eqnarray*}
Lining up similar terms, we have
$   r   = \rho + \alpha g - V_m/ 2 $ and
\begin{eqnarray*}
%    r   \;\;=\;\; \rho + \alpha g - V_m/ 2 \\
    (\alpha d_1^\top + b^\top)  A
        \;\;=\;\;  \tau b^\top + d_2^\top
        \;\;\;\;\Rightarrow \;\;\;\;
        b^\top \;\;=\;\; (\alpha d_1^\top A - d_2^\top) (\tau I -  A )^{-1} .
%        \label{eq:macrofin-solution}
\end{eqnarray*}
As before, this gives us a unique stationary solution
under the stated conditions:  $ A $ stable and $\tau>1$.


Now consider identification.
Suppose we observe the state $x_t$, the interest rate $i_t$,
the inflation rate $\pi_t$, and log consumption growth $g_t$,
but not the shock $s_{2t}$ to the Taylor rule.
From observations of the state,
we can estimate the autoregressive matrix $ A $,
%the covariance matrix $C C^\top$,
and from observations of consumption growth we can estimate
the shock coefficients $d_1$.
We can also estimate $a$ and $b$ by projecting
$i_t$ and $\pi_t$ on the state.
With $a^\top = (\alpha d_1^\top + b^\top)  A $ known, that leaves us to solve
\begin{eqnarray}
    a^\top
        &=&  \tau b^\top + d_2^\top
        \label{eq:macrofin-solution-2}
\end{eqnarray}
for the Taylor rule's inflation parameter $\tau$  and shock coefficients $d_2$:
$n$ equations in the $n+1$ unknowns $(\tau,d_2)$.
The identification problem is the same as Section \ref{sec:examples}:
without further restrictions,
the Taylor rule is not identified.
This is Cochrane's conclusion in somewhat more general form.

We can, however, identify the monetary policy rule
if we place one or more restrictions on the shock coefficients $d_2$.
One such case was mentioned earlier:
choose $d_1$ and $d_2$ so that the shocks $s_{1t}$ and $s_{2t}$ are independent.
We'll return to this shortly.
Another example is a zero in the vector $d_2$ ---
what is traditionally termed an exclusion restriction.
Suppose the $i$th element of $d_2$ is zero.
Then the $i$th element of (\ref{eq:macrofin-solution-2}) is
\begin{eqnarray*}
    a_i &=& \tau b_i .
\end{eqnarray*}
If $b_i \neq 0$ (a regularity condition we'll come back to),
this determines $\tau$.
Given $\tau$, and our estimates of $a$ and $b$,
we can now solve (\ref{eq:macrofin-solution-2}) for
the remaining components of $d_2$.
We can do the same thing with restrictions based on linear combinations.
Suppose $d_2^\top e = 0$ for some known vector $e$.
Then we find $\tau$ from $a^\top e = \tau b^\top e$.
Any such linear restriction on the shock coefficient $d_2$ allows
us to identify the Taylor rule.

Cochrane's example is a special case with shocks to
consumption growth turned off:  $d_1 = 0$.
As a result, all the variation in inflation and the short rate
comes from monetary policy shocks $s_{2t}$.
Special case or not, the conclusion is the same:
we need one restriction on $d_2$ to identify the (one) Taylor rule parameter $\tau$.
Note well:  The restriction applies to the Taylor rule shock,
and does not require a shock in the other equation.


\subsection{An exponential-affine model}

We take a similar approach to an exponential-affine model,
adding a Taylor rule to an otherwise standard bond-pricing model.
The model consists of a real pricing kernel,
a Taylor rule, and the law of motion (\ref{eq:state-dynamics}) for the state.
The first two are
\begin{eqnarray*}
    m_{t+1} &=& -\rho - s_{1t} + \lambda^\top w_{t+1} \\
    i_t &=& r + \tau \pi_t + s_{2t} .
\end{eqnarray*}
As usual, the shocks are linear functions of the state:  $s_{it} = d_i^\top x_t$.
This model differs from the example in Section \ref{sec:example-affine}
in having a Taylor rule as well as a bond pricing relation.

We solve the model by the usual method.
Given a guess $\pi_t = b^\top x_t$ for inflation,
the nominal pricing kernel is
\begin{eqnarray*}
    m^{\$}_{t+1} &=& m_{t+1} - \pi_{t+1}
            \;\;=\;\; - \rho - (d_1^\top + b^\top  A ) x_t + (\lambda^\top - b^\top C) w_{t+1} .
\end{eqnarray*}
The short rate follows from (\ref{eq:example2-shortrate}):
\begin{eqnarray*}
    i_t &=& \rho - V_m/2 + (d_1^\top + b^\top  A )^\top x_t ,
\end{eqnarray*}
where $V_m = (\lambda^\top - b^\top C)(\lambda - C^\top b)$.
Equating this to the Taylor rule gives us
$ r = \rho - V_m/2$ and
\begin{eqnarray*}
    d_1^\top + b^\top  A  &=& \tau b^\top + d_2^\top
    \;\;\;\Rightarrow\;\;\;
    b^\top \;\;=\;\; (d_1^\top - d_2^\top) (\tau I -  A )^{-1} .
\end{eqnarray*}
This is the unique stationary solution for $b$ under the usual conditions.

Identification follows the same logic as the preceding example.
Let us say, again, that we observe the state $x_t$, the short rate $i_t$,
and inflation $\pi_t$.
From the latter we estimate $a = d_1^\top + b^\top A$ and $b$.
That gives us $n$ equations in the $n+1$ unknowns $(\tau, d_2)$.
The model is identified only when we impose one or more
restrictions on the coefficient vector $d_2$ of the monetary policy shock.
If, for example, the $i$th element of $d_2$ is zero,
then $\tau$ follows from $a_i = \tau b_i$ as long as $b_i \neq 0$.

This model is a generalization of the previous one in which we've given
the real pricing kernel a more flexible structure.
Evidently the structure of the pricing kernel has little bearing on identification.
We need instead more structure on the shock to the Taylor rule to compensate
for not observing it.


\subsection{Discussion}
\label{sec:macro-fin-discussion}

We have seen that one restriction suffices to identify the Taylor rule
in these examples.
The question is why this works.
Is it similar to the use of exclusion restrictions in simultaneous equations models?
Most econometrics textbooks illustrate exclusions of this kind with supply and demand.
There we need a variable in one equation that's missing --- excluded ---
from the other.
To identify the demand equation,
we need a variable in the supply equation that's excluded from demand.
That's not the case here.
We can identify the Taylor rule even when there are no shocks in
the other equation if we have a restriction on the Taylor rule shock.
The issue is not whether we have the right
configuration of shocks across equations, but whether we observe them.
When we don't observe the shock to the Taylor rule,
we need additional structure in the same equation to deduce its parameters.

The same logic applies to Gertler's example in Section \ref{sec:examples-discussion},
where we used independence of the two shocks to identify the Taylor rule.
Doesn't that involve shocks in the second equation?
Well, yes, but the critical feature of independence here
is the restriction it places on the Taylor rule shock.
%Let us denote the covariance matrix of the state by $V_x$.
The shocks are uncorrelated, hence independent, if $d_2^\top V_x d_1 = 0$.
But that's a linear restriction $d_2^\top e = 0$ on the coefficient vector $d_2$ of the Taylor rule shock.  In this case, $e = V_x d_1$.
The same holds for restrictions on innovations to the shocks.
They're independent and uncorrelated if $d_2^\top V_w d_1 = 0$.
In the representative agent model of Section \ref{sec:rep-agent},
such a restriction is easily implemented.
If we observe consumption growth (\ref{eq:con-growth}),
then we also observe $s_{1t}$ and can use it to estimate $d_1$
and compute the restriction on $d_2$,
the coefficient vector of the Taylor rule shock.
We give numerical examples in Appendix \ref{app:examples}.
These restrictions have no particular economic rationale in this case,
but they illustrate how independence works.
Similar ``orthogonality conditions'' appear throughout the New Keynesian literature.

A similar question arises with restrictions on interest rate coefficients.
Suppose we know that a linear combination of interest rate coefficients is zero:
 $a^\top e = 0$ for some known $e$.
Then (\ref{eq:macrofin-solution-2}) gives us a restriction connecting
the Taylor rule shock and its coefficient vector:
$ \tau b^\top e + d^\top e = 0$.
One interpretation is that we've used a restriction from another part of the model
for identification.
We would say instead that any such restriction on interest rate behavior
implies a restriction on the Taylor rule, which identifies the policy rule for the usual reasons.

%  ?? add regularity condition here, show what you need to make sure bi \neq 0.
%  that sets up that this is a system result, not a single equation

Another difference from traditional simultaneous equation methods
is that single-equation estimation methods generally won't work.
We need information about the whole model to deduce the Taylor rule.
In the model of Section \ref{sec:rep-agent}, for example,
we need to estimate an interest rate equation to find $a$ and
an inflation equation to find $b$, before applying (\ref{eq:macrofin-solution-2}) to find $\tau$.
This reflects what Hansen and Sargent (1980, page 37) call the ``hallmark''
of rational expectations models:  cross-equation restrictions
connect the parameters in one equation to the parameters in the others.


\section{A model with a Phillips curve}
\label{sec:phillips-curve}

We apply the same logic to an example with a stronger
New Keynesian flavor.
We add a Phillips curve to the model and an output gap to the Taylor rule.
Models with similar features are described by
Carrillo, Feve, and Matheron (2007),
Canova and Sala (2009),
Christiano, Eichenbaum, and Evans (2005),
Clarida (2001),
Clarida, Gali, and Gertler (1999),
Cochrane (2011),
Gali (2008),
King (2000),
%Nason and Smith (2008),
Shapiro (2008),
Woodford (2003), and many others.
Our contribution is a modest one:
to add an asset pricing relation,
which we think of as an illustration of how asset prices
might be introduced into these models.

Despite the additional economic structure,
the logic for identification is the same:
we need restrictions on the shock coefficients
to identify the Taylor rule.
What changes is that we need two restrictions,
one for each of the two parameters of the rule.
We face similar issues in identifying the Phillips curve.
If its shock isn't observed,
we need restrictions to identify its parameters.


Our model consists of equations (\ref{eq:macrofin-euler})
and (\ref{eq:macrofin-real-m}) plus
\begin{eqnarray*}
%    m_{t} &=& -\rho - \alpha g_t  \\
        \pi_t &=&  \beta E_t \pi_{t+1} + \kappa g_t + s_{1t} \\
    i_t &=& r + \tau_1 \pi_t + \tau_2 g_t + s_{2t}.
\end{eqnarray*}
The first equation is a New Keynesian Phillips curve and
the second is a Taylor rule, which now includes an output term.
In addition, we have
the law of motion (\ref{eq:state-dynamics}) for the state
and the shocks $s_{it} = d_i^\top x_t$ for $i=1,2$.

We now have a two-dimensional rational expectations model
in the forward-looking variables $\pi_t$ and $g_t$.
The solution of such models is described in Appendix \ref{app:hs-formulas},
but we can illustrate the challenges of identification
with our usual guess-and-verify procedure.
We guess a solution that includes
$\pi_t = b^\top x_t$ and $g_t = c^\top x_t$.
Then the pricing relation gives us
\begin{eqnarray*}
        i_t &=& \rho - V_m/2 + a^\top x_t %\\
%        a^\top &=&  (\alpha c^\top + b^\top)  A  \\
%        V_m &=& a^\top C C^\top a.
\end{eqnarray*}
with
$  a^\top =  (\alpha c^\top + b^\top)  A  $
and $  V_m = a^\top C C^\top a $.
If we equate this to the Taylor rule and collect terms,
we have $ r = \rho - V_m/2 $ and
\begin{eqnarray}
    a^\top &=& \tau_1 b^\top + \tau_2 c^\top + d_2^\top .
    \label{eq:i-pc}
\end{eqnarray}
The Phillips curve implies
\begin{eqnarray}
    b^\top &=& \beta  A  b^\top + \kappa c^\top + d_1^\top .
    \label{eq:pi-pc}
\end{eqnarray}
As others have noted, the conditions for this to have a unique stationary solution
are more stringent than before.
We'll assume that they're satisfied.


Let's turn to identification.
Suppose we, the economists, observe the state $x_t$, the interest rate $i_t$,
the inflation rate $\pi_t$, and log consumption growth $g_t$.
From them, we can estimate the autoregressive matrix $ A $
and the coefficient vectors $(a,b,c)$.
In the Taylor rule, the unknowns are the policy parameters $(\tau_1,\tau_2)$
and the coefficient vector $d_2$ for the shock.
If we observe the shock $s_{2t}$,
equation (\ref{eq:i-pc}) gives us $n$ equations
to solve for $\tau_1$ and $\tau_2$.
As long as the dimension $n$ of the state is at least two, the Taylor rule
parameters are identified.
If we do not observe the shock, then we need two restrictions on its
coefficient vector $d_2$.
The conclusion is the same as before,
but with two parameters to identify we need two restrictions
on the vector of shock coefficients.

The same logic applies to identifying the parameters of the Phillips curve.
If we observe the shock $s_{1t}$,
equation (\ref{eq:pi-pc}) gives us $n$ equations to solve for
the parameters $\beta$ and $\kappa$.
If we do not observe its shock $s_{1t}$, then two restrictions are needed
to identify its two parameters.
The identification problem for the Phillips curve has the same structure as
the Taylor rule, although in practice they've been treated separately.
For more on the subject,
see the extensive discussions in Canova and Sala (2009),
Gali and Gertler (1999),
Nason and Smith (2008), and Shapiro (2008).

Standard implementations of New Keynesian models typically assume
independent AR(1) shocks.
See, for example, Gali (2008, ch 3).
In our framework, this amounts to $n-1$ zero restrictions on the coefficient vectors $d_i$.
That's generally sufficient to identify the structural parameters of the model,
including those of the Taylor rule,
but more restrictive than necessary.
With respect to the Taylor rule parameter,
each element $i$ for which $d_{2i}=0$ leads,
via equation (\ref{eq:i-pc}),
to an equation of the form $a_i = \tau_1 b_i + \tau_2 c_i$.
As long as $(a_i,b_i,c_i) \neq (0,0,0)$, any two such equations will
identify the Taylor rule parameters $(\tau_1,\tau_2)$.
Similar logic applies to the Phillips curve.


\section{Observing the state}
\label{sec:observability}

We have seen, in a number of examples, that to identify the parameter(s) of the Taylor rule,
we need information about its shock.
If we do not observe the shock, then we must impose restrictions on its form.

Our approach so far is predicated on observing the state.
What happens if we observe the state indirectly?
Or a noisy signal of the state?
We show that neither changes the nature of the identification problem.


\subsection{Indirect observation of the state}

In many applications, the state variable is latent:
we don't observe it directly,
but we may be able to infer something about it.
Examples include dynamic factor models,
exponential-affine bond-pricing models,
and structural vector autogressions.

In some of these models, the state is not completely determined:
linear transformations of the state are observationally equivalent.
Consider a model based on the linear law of motion (\ref{eq:state-dynamics}).
If the state $x_t$ is not observed directly, then it is indistinguishable
from a model with state $\hat{x}_t = T x_t$,
where $T$ is an arbitrary square matrix of full rank.
The law of motion is
\begin{eqnarray}
    \hat{x}_{t+1} &=& T A T^{-1} \hat{x}_t + T C w_{t+1}
            \;\;=\;\; \widehat{A} \hat{x}_t + \widehat{C} w_{t+1} .
    \label{eq:state-dynamics-transformed}
\end{eqnarray}
where $\widehat{A} = T A T^{-1}$ and $\widehat{C} = T C$.
By the same logic,
the shocks are $ s_{it} = \hat{d}_i^{\top} \hat{x}_t$ with
$ \hat{d}_i^{\top} = {d}_i^\top T^{-1} $.


We can transform the rest of the model the same way.
Consider the representative agent model described in Section \ref{sec:rep-agent}.
For any choice of state $\hat{x}_t$,
we can estimate the associated parameter vectors for the observables.
The short rate is then related to the state by
$ i_t = r + \hat{a}^{\top} \hat{x}_t$
with $ \hat{a}^{\top } = a^\top T^{-1}$.
Similarly, inflation is related by
$\pi_t = \hat{b}^{\top} \hat{x}_t$ with $ \hat{b}^{\top} = b^\top T^{-1}$.
The Taylor rule then implies
\begin{eqnarray*}
    \hat{a}^{\top} &=& \tau \hat{b}^{\top} + \hat{d}_2^{\top} ,
%    \label{eq:macrofin-solution-2-xprime}
\end{eqnarray*}
the analog of equation (\ref{eq:macrofin-solution-2}) for the transformed state.

Consider the problem of identifying $\tau$ when we observe
the transformed state $\hat{x}_t$ but not the original state $x_t$
or the transforming matrix $T$.
The identification problem is the same as before:
we need one restriction on $ \hat{d}_2$ to identify
the single Taylor rule parameter $\tau$.
The only question is whether the restrictions we place on $d_2$
are intelligible when we translate them to $\hat{d}_2$.
Consider a general linear restriction $d_2^{\top} e =0,$
where at least one element of $e$ is non-zero.
This restriction can be expressed  $\hat{d}_2^{\top} T e =0$,
so the restricting vector is $\hat{e}=Te$.
But if we don't know $T$, can we deduce $\hat{e}$?

There are at least two cases where the restrictions
translate naturally to the transformed state.
These cases have nearly opposite economic interpretations,
so they offer a range of choices that can lead to identification.

In the first case, suppose the Taylor rule shock is uncorrelated with the other shock.
Such ``orthogonality conditions'' are standard in the New Keynesian literature,
where most shocks are assumed to be independent of the others.
%Cochrane (2011, page 595) notes that it's common to assume that the Taylor rule
%shock have ``sufficient orthogonality properties to allow some estimation.''
We saw an example in Section \ref{sec:macro-fin-discussion}.
In terms of the original state $x_t$,
the restriction takes the form $ d_2^\top V_x d_1 = 0$.
In terms of the transformed state $\hat{x}_t$, we have
\begin{eqnarray*}
    \hat{d}_2^{\top} E \big( \hat{x}  \hat{x}^{\top} \big) \hat{d}_1
        &=&  (d_2^\top T^{-1}) (T V_x T^\top) (d_1^\top T^{-1})^\top
        \;\;=\;\; d_2^\top V_x d_1 .
\end{eqnarray*}
It's clear that this restriction is invariant to linear transformations of the state.
We show how this might work in practice in Appendix \ref{app:examples}.

In the second case, optimal monetary policy dictates a connection between
shocks $s_{1t}$ and $s_{2t}$.
See, for example, Gali (2008) and Woodford (2003).  % chapters??
When a monetary authority minimizes an objective function,
all variables of interest are affected by by $s_{1t}$.
As a result, an optimal policy rule will make $s_{2t}$ proportional to $s_{1t}$.
If we express this by setting $s_{2t} = k s_{1t}$ for some constant $k$,
it implies the restriction $ d_2^\top - k d_1^\top = 0$ in terms of the original
state variable $x_t$.
In terms of the transformed state $\hat{x}_t$, the restriction is
\begin{eqnarray*}
    \hat{d}_2^{\top} - k \hat{d}_1^{\top}
            &=& d_2^\top T^{-1}- k d_1^\top T^{-1}
            \;\;=\;\; (d_2^\top - k d_1^\top) T^{-1}
            \;\;=\;\; 0,
\end{eqnarray*}
which is independent of the transformation $T$.

These restrictions apply to a given transformed state $\hat{x}_t$.
One way to determine $\hat{x}_t$ is to choose a set of observable variables of the same dimension.
We give some examples in Appendix \ref{app:examples}.
Another is to select a specific transformation.
This typically involves restrictions on both $C$ and $A$.
For the former, a common example is $ T = C^{-1}$,
which gives each state equation an independent innovation with unit variance.
For the latter, the typical approach is to consider rotations:
transformation by orthogonal matrices.
Rotations that deliver lower-triangular $\widehat{A}$ are a common choice,
but not a necessary one.
Variants of this approach are used in
dynamic factor models (Bai and Wang, 2012, and Bernanke, Boivin, and Eliasz, 2005),
exponential-affine term structure models (Joslin, Singleton, and Zhu, 2011),
and structural vector autoregressions
(Leeper, Sims, and Zha, 1996, and Watson, 1994).


\subsection{Noisy observation of the state}

In other cases we may observe the state with noise,
even after appropriate transformation.
One example is a state with dimension $n>2$
with observations of only the two variables $i_t$ and $\pi_t$.
Another is a larger collection of observables
in which each variable is measured with error.

The Kalman filter deals with precisely this problem.
Here we have three sets of variables.
The state $x_t$ follows
the usual linear law of motion (\ref{eq:state-dynamics}).
Endogenous variables $z_t$ are related to the state by $ z_t = B x_t$,
where each row of $B$ represents one of the model's endogenous variables.
In our examples, $z_t$ includes the one-period interest rate $i_t$,
the inflation rate $\pi_t$, and so on.
Finally, we have a set of variables $y_t$ that we observe,
which can include some or all of $z_t$.

A state-space model adds a measurement equation for the observables
 to the law of motion for the state:
\begin{eqnarray}
    y_t &=& G x_t + H v_t .
        \label{eq:meas}
\end{eqnarray}
We posit measurement errors $v_t \sim \mbox{NID}(0,I)$ that are independent of $w_t$,
but could easily incorporate arbitrary correlation between them.
The observables are likely to include at least some of the endogenous
variables, including those mentioned above,
and possibly some of the state variables, too.
In modern ``data-rich'' applications, the dimension of $y_t$ is large
relative to that of $x_t$,
but the filter works even in the opposite situation.

Our question becomes how well observations
of $y_t$ substitute for observations of the state $x_t$.
The question is a classical one and has a standard answer
based on the Kalman filter:
use $y_t$ to estimate the state $x_t$ and proceed as before
using the estimate in place of the state.
The Kalman filter is widely used in macroeconomics,
so our treatment will be brief.
Some of the essentials are given in Appendix \ref{app:kalman}.

Two conditions guarantee that the state is, for our purposes, observed.
First, we need the law of motion (\ref{eq:state-dynamics})
to generate a state of dimension $n$, not a lower dimensional subspace.
It's sufficient to assume that $C$ has rank $n$,
but when (\ref{eq:state-dynamics}) is in companion form that's not realistic.
In formal terms, we need $(A,C)$ to be controllable,
so that we can attain any $x_t$ in $\mathbb{R}^n$ with some realization of disturbances $w_t$.
If that's not the case, we should reduce the dimension of our problem.
Second, and more critically, we need $(A,G)$ to be observable;
for the matrix
\begin{eqnarray*}
    \mathcal{O} &=&
    \left[
    \begin{array}{c}
        G \\ G  A  \\ \vdots \\ G  A ^{n-1}
    \end{array}
    \right]
\end{eqnarray*}
to have rank $n$.
Roughly speaking: the state $x_t$ must generate enough variation in the observed variables $y_t$
that we can (approximately) reverse engineer the state.
With these two conditions, we can estimate the state
from observations of the infinite history $y^t = (y_t, y_{t-1}, \ldots)$.

The state typically isn't recovered exactly, but the Kalman filter generates
recursive estimates from the history of observables, which we might represent by
\begin{eqnarray*}
    {x}_{t|t} &=& E \left( x_t | y^{t} \right).
\end{eqnarray*}
The state itself therefore includes an error:  $ x_t = {x}_{t|t} + \varepsilon_t $
with the error $\varepsilon_t$ orthogonal to the history $y^{t}$.

Now think about how we use the state:  we need it to
estimate the coefficient vectors tying endogenous variables to the state.
Consider a specific endogenous variable $z_{it}$ connected to the state by
$ z_{it} = b^\top x_t $, where $b^\top$ is the $i$th row of $B$.
The inflation rate is a good example.
Then
\begin{eqnarray*}
    z_{it}  &=& b^\top x_t
        \;\;=\;\; b^\top {x}_{t|t} + b^\top \varepsilon_t .
\end{eqnarray*}
If $z_{it}$ is one of the observables, then the last term is noise, orthogonal to the history $y^t$
and therefore to $x_{t|t}$.
We can estimate $b$ by projecting $z_{it}$ on $x_{t|t}$. % rather than $x_t$.
%This is just a projection on a smaller information set than $x_t$.
By doing this for each observable variable,
we can estimate coefficient vectors as before
using ${x}_{t|t}$ in place of $x_t$.

%This seems to us to echo Sims (1998, p 935),
% who argues that ``models can disagree on ... shocks while agreeing on their effects.''


\subsection{Discussion}
\label{sec:state-discussion}

The challenges of observing the state raise a number of questions.
One is how to observe the state in practice.
In some cases, we may find that the state is spanned by
some collection of observables.
We think financial variables and forecasts are likely to be helpful here.

Models that have an exponential-affine structure,
including all of the models in this paper,
have a structure that makes forward rates natural state variables.
If $q_t^h$ is the price at date $t$ if a claim to one dollar at $t+h$,
then continuously-compounded forward rates are defined by
$ f^h_t = \log (q^h_t/q^{h+1}_t)$.
The short rate is $i_t = f^0_t$.
In  our examples, the short rate takes the form (ignoring the intercept) $i_t = a^\top x_t$
and forward rates are $f^h_t = a^{\top} A^h x_t$.
The vector $f_t$ of the first $h$ forward rates has the form
\begin{eqnarray*}
    f_t  \;\;=\;\; \left[  \begin{array}{c} f^0_t \\ f^1_t \\ \vdots \\ f^{h-1}_t \end{array} \right]
        &=&
    \left[  \begin{array}{c} a^\top \\ a^\top A \\ \vdots \\ a^\top A^{h-1} \end{array} \right]
     x_t
    \;\;=\;\; U x_t .
\end{eqnarray*}
You may notice a resemblance to the observability condition here,
in this case the observability of $(A, a^\top)$.
If the dimension of the state and the number of forward rates are the same,
and $U$ is invertible,
we find $x_t$ by inverting the mapping from states to forward rates:  $x_t = U^{-1} f_t$.
If not, we can use them as inputs to the Kalman filter.
Yields, defined by $ i^h_t = h^{-1} \sum_{j=1}^h f^{j-1}_t $,
are linear functions of forward rates, so they can be used the same way.

Forecasts have a similar mathematical structure.
Even noisy measurements of conditional expectations can be helpful.
Consider the forecast of variable $z_{it}$ $h$ periods ahead.
If $ z_{it} = b^\top x_t$ for some coefficient vector $b$,
then a forecast might be expressed
\begin{eqnarray*}
    F_t z_{i,t+h}&=& E_t z_{i,t+h} + u_{t,t+h}
            \;\;=\;\; b^\top A^h x_{t+h} + u_{t,t+h},
\end{eqnarray*}
with some distribution of noise $u_{t,t+h}$.
Surveys provide forecasts for a range of $h$'s,
each of which can be a useful input into a Kalman filter estimate of the state.
Chernov and Mueller (2012), Chun (2011), and Kim and Orphanides (2012)
are examples that use survey forecasts in state-space frameworks.

Another question is whether we can proceed if the observability condition fails,
leaving us with hidden states.
In some cases, it may be possible to proceed with the observable subspace of the state.
Suppose, for example, that $x_{t|t}$ has dimension $k < n$.
We can still project endogenous variables on this estimate of the state
and use these projections to estimate the Taylor rule.
We simply need restrictions on the projection of the Taylor rule shock
onto this subspace.
The number of restrictions needed for identification is the same:
one in the examples of Section \ref{sec:macrofin}, two in the example of Section \ref{sec:phillips-curve}.

\begin{comment}
\begin{itemize}
\item Bernanke and Blinder (1992, p 902):
``Two types of identifying assumptions are most obvious.
[The first] means assuming that there is no feedback from the economy
to policy actions within the period. ...
[The second] is to suppose ... that policy actions affect real variables
only with a lag.''

\item Bernanke and Mihov (1998, p 874):
``[I]t is sufficient to assume that policy shocks do not affect
the given macro variables within the current period.''

\item Sims (1998, p 933):
``A reasonable picture of the effects of monetary policy shifts
emerges only under the identifying assumptions of delay in the reaction
of certain `sluggish' private sector variables to monetary policy shifts.''
\end{itemize}

These are zeros in the dependence of some variables on the state
(the current and lagged values of all the variables in the system),
so they fit into our framework.
But some concern the policy shock, others concern the behavior
of other variables.

\end{comment}



\section{Conclusion}

Identification is always an issue in applied economic work,
perhaps nowhere more so than in the study of monetary policy.
That's still true.
We have shown, however, that
(i)~the problem of identifying
the systematic component of monetary policy (the Taylor rule parameters)
in macro-finance models stems from our inability to observe the nonsystematic component
(the shock to the rule)
and (ii)~the solution is to impose restrictions on the shock.
We are left where we often are in matters of identification:
trying to decide which restrictions are plausible,
and which are not.


%\begin{quote}
%It may be that the evidence used to select the final hypothesis
%from the subclass can be consistent with at most one hypothesis in it,
%in which case the ``model'' is said to be ``identified.''
%But this way of describing the concept of ``identification'' is essentially
%a special case of the more general problem of selecting among alternative
%hypotheses consistent with the evidence --- a problem
%that must be decided by some such arbitrary principle as Occam's razor.
%\end{quote}
%



%\end{document}
% ***********************************************************************
\pagebreak
%\baselineskip=\oldbaselineskip
\appendix
\section{Solution of rational expectations models}
\label{app:hs-formulas}

%We show how to solve linear rational expectations models.
%We consider models with a single forward-looking variable first, then
%move to multi-dimensional systems.

Consider the class of forward-looking linear rational expectations models,
\begin{eqnarray*}
         z_t     &=& \Lambda E_t z_{t+1} + D x_t \\
        x_{t+1}  &=& A x_t + C w_{t+1} .
\end{eqnarray*}
Here $x_t$ is the state,
$\Lambda$ is stable (eigenvalues less than one in absolute value),
$A$ is also stable,
and  $w_t \sim \mbox{NID}(0,I)$.
The goal is to solve the model and link $z_t$ to the state $x_t$.

%\subsection{One-dimensional models}

{\it One-dimensional case.\/}
If $z_t$ is a scalar and the shock is $s_t = d^\top x_t$, we have
\begin{eqnarray}
    z_t &=&    \lambda E_t z_{t+1} + d^\top x_{t} .
    \label{eq:for-diff-eq-scalar}
\end{eqnarray}
%The solution has the form $z_t = b^\top x_t$ for some coefficient vector $b$.
Substituting repeatedly gives us
\begin{eqnarray*}
    z_t &=&  \sum_{j=0}^\infty \lambda^j  d^\top E_t x_{t+j}
        \;\;=\;\;  d^\top \sum_{j=0}^\infty \lambda^j  A^j x_{t}
        \;\;=\;\;  d^\top (I-\lambda A)^{-1} x_t.
\end{eqnarray*}
The last step follows from the matrix geometric series if $A$ is stable and $|\lambda|<1$.
Under these conditions, this is the unique stationary solution.


The same solution follows from the method of undetermined coefficients,
but the rationale for stability is less clear.
We guess $ z_t = b^\top x_t$ for some vector $b$.
Then the difference equation tells us
\begin{eqnarray*}
    b^\top x_t &=&  b^\top \lambda A x_t + d^\top x_t .
\end{eqnarray*}
Collecting terms in $x_t$ gives us
$ b^\top =  d^\top (I-\lambda A)^{-1} $.
%What's missing is the logic for the stability conditions on $A$ and $\lambda$.

This model is close enough to the examples of Sections \ref{sec:examples} and \ref{sec:macrofin}
that we can illustrate their identification issues in a more abstract setting.
Suppose we observe the state $x_t$ and the endogenous variable $z_t$,
but not the shock $s_t$.
Then we can estimate $A$ and $b$.
Equation (\ref{eq:for-diff-eq-scalar}) then gives us
\begin{eqnarray*}
    b^\top &=& \lambda b^\top A + d^\top .
\end{eqnarray*}
If $x$ has dimension $n$, we have $n$ equations in the
$n+1$ unknowns $(\lambda, d)$;
we need one restriction on $d$ to identify the
parameter $\lambda$.


{\it Multi-dimensional case.\/}
If $z_t$ is a vector, as in Section \ref{sec:phillips-curve},
we have
\begin{eqnarray*}
    z_t &=&    \Lambda E_t z_{t+1} + D^\top x_{t} .
\end{eqnarray*}
Repeated substitution gives us
\begin{eqnarray*}
    z_t &=&  \sum_{j=0}^\infty \Lambda^j D A^j x_{t} .
\end{eqnarray*}
That gives us the solution $ z_t = B x_t$ where
\begin{eqnarray*}
    B &=& D + \Lambda B A \;\;=\;\; \sum_{j=0}^\infty \Lambda^j D A^j .
\end{eqnarray*}
The solution is
\begin{eqnarray*}
    \mbox{vec}(B) &=& (I - A^\top \otimes \Lambda)^{-1} \mbox{vec} (D) .
\end{eqnarray*}
See, for example,
Anderson, Hansen, McGrattan, and Sargent (1996, Section 6) or Klein (2000, Appendix B).
The same sources also explain how to solve
rational expectations models with endogenous state variables.


%\subsection{Models with predetermined variables}
%
%Coming...
%This is more of a mess, because we need to separate the stable and unstable
%components of the model and connect
%them to


\section{Numerical examples}
\label{app:examples}

We illustrate some of the issues raised in the paper with numerical examples
of the model in Section \ref{sec:rep-agent}:
a representative agent with power utility and given consumption growth.
We show how identification works when
we observe the state
and when we observe only a linear transformation of the state.
%and (iii)~infer the state from other observables.
In each case we use an orthogonality restriction
on the shock to the Taylor rule.

We give the model a two-dimensional state and
and choose parameter values
$\tau=1.5$, $\alpha=5$, and
\begin{eqnarray*}
    A &=&
        \left[
        \begin{array}{rr}
        0 & 1 \\ -0.05 & 0.9
        \end{array}
        \right],
        \;\;\;\;\;
    C \;\;=\;\;
        \left[
        \begin{array}{rr}
        0.0078 & 0 \\ -0.0004 & 0.0003
        \end{array}
        \right] .
\end{eqnarray*}
The consumption growth shock is governed by $ d_1^\top = (1, 0)$.
The monetary policy shock is $ d_2^\top = (\delta, 1)$,
with $\delta$ chosen to make $s_2$ uncorrelated with $s_1$.
These inputs imply
\begin{eqnarray*}
    V_x &=&
       \left[
        \begin{array}{rr}
        0.6432 & -0.0069 \\ -0.0069 & 0.0258
        \end{array}
        \right] \cdot 10^{-4} ,
\end{eqnarray*}
so $ d_2^\top V_x d_1 = 0 $ implies $\delta = 0.0108$.

These are the inputs, the source of data that we can use
to estimate the Taylor rule.
The question is whether we can do that under different
assumptions about observability of the state.

{\it State observed.\/}
Suppose, first, that we observe the state $x_t$.
Then we can use observations of the interest rate $i_t$ and inflation rate $\pi_t$ to
recover the coefficient vectors
\begin{eqnarray*}
    a &=&
    \left[
        \begin{array}{r}
        -0.3152 \\ 10.4566
        \end{array}
    \right] ,
    \;\;\;\;\;
    b \;\;=\;\;
    \left[
        \begin{array}{r}
        -0.2174 \\ 6.3044
        \end{array}
    \right] .
\end{eqnarray*}
Similarly, observations of consumption growth allow us to recover $d_1$.
We do not observe the Taylor rule shock $s_{2t}$,
so its coefficient vector $d_2$ remains unknown.
A least squares estimate of the Taylor rule here gives us $\tau = 1.6510$
which, of course, isn't the value that generates the data.


Identification requires a restriction on $d_2$.
We know $d_2$ satisfies the
orthogonality condition $d_2^\top e = 0$ with $e = V_x d_1$.
With our numbers, $ e^\top = (0.6364, -0.0069 ) \cdot 10^{-4}$.
We post-multiply (\ref{eq:macrofin-solution-2}) by $e$ to get
$ a^\top e = \tau b^\top e $.
This implies $\tau = 1.5$, the value we started with.
We can now recover $d_2$ from the same equation:
$ d_2^\top = a^\top - \tau b^\top = (0.0108, 1.0000)$.

This is simply the procedure we outlined in
Section \ref{sec:rep-agent}, but it gives us a concrete
basis of comparison for situations in which we don't directly
observe the state.

{\it State observed indirectly.\/}
Now suppose we don't observe the state, but we observe enough variables
to deduce a linear transformation of the state.
We consider two examples.

In our first example, we observe the interest rate and inflation rate
and use them as our transformed state: $\hat{x}_t = (i_t,\pi_t)^\top$.
Then $\hat{x}_t = T x_t$ with
\begin{eqnarray*}
    T &=&
        \left[
        \begin{array}{c}
        a^\top \\ b^\top
        \end{array}
        \right] .
\end{eqnarray*}
This has something of the flavor of a structural vector autoregression,
albeit a simple one.
The law of motion for the transformed state is equation (\ref{eq:state-dynamics-transformed})
with
\begin{eqnarray*}
    \widehat{A} &=&
       \left[
        \begin{array}{rr}
         -4.6170 &   9.1006 \\  -2.8044  &  5.5170
        \end{array}
        \right] ,
%        \;\;\;\;\;
%    \widehat{C} \;\;=\;\;
%        \left[
%        \begin{array}{rr}
%         -0.0065 &   0.0036 \\  -0.0042  &  0.0022
%        \end{array}
%        \right] .
\end{eqnarray*}
which is easily estimated.

Now consider identification.
If we regress $i_t$, $\pi_t$, and log consumption growth on $\hat{x}_t$,
we get the coefficient vectors
\begin{eqnarray*}
    \hat{a} &=&
    \left[
        \begin{array}{r}
        1 \\ 0
        \end{array}
    \right] ,
    \;\;\;\;\;
    \hat{b} \;\;=\;\;
    \left[
        \begin{array}{r}
        0 \\ 1
        \end{array}
    \right] ,
        \;\;\;\;\;
    \hat{d}_1 \;\;=\;\;
    \left[
        \begin{array}{r}
         22.07 \\  -36.60
        \end{array}
    \right] .
\end{eqnarray*}
From observations of $\hat{x}_t$, we can estimate its covariance matrix
\begin{eqnarray*}
    V_{\hat{x}} &=&
       \left[
        \begin{array}{rr}
         0.2932 &   0.1775 \\  0.1775  &  0.1075
        \end{array}
        \right] \cdot 10^{-3} .
\end{eqnarray*}
Finally, the orthogonality condition in these coordinates
is $\hat{d}_2^{\top} \hat{e} = 0$ with $\hat{e} = V_{\hat{x}} \hat{d}_1 $.
With our numbers, we have $\hat{e}^\top = (-0.2718, -0.1812) \cdot 10^{-4}$.
%
As before, we apply the orthogonality condition to
equation (\ref{eq:macrofin-solution-2}), which gives us
$\tau = \hat{a}^{\top} \hat{e} / \hat{b}^{\top} \hat{e} = 1.5 $,
the number we started with.
It is clear from this that we are still able to recover the Taylor rule
from this linear transformation of the state.


In our second example, we use the first two forward rates as the state:
$\hat{x}_t = (f_t^0=i_t, f_t^1)^\top$.
As we saw in Section \ref{sec:state-discussion},
forward rates are connected to the original state $x_t$ by
$\hat{x}_t = T x_t$, where
\begin{eqnarray*}
    T &=&
        \left[
        \begin{array}{c}
        a^\top \\ a^\top A
        \end{array}
        \right] .
\end{eqnarray*}
The same series of calculations gives us $\tau$.



\section{Elements of Kalman filtering}
\label{app:kalman}

We outline some of the essential elements of Kalman filtering.
Hansen and Sargent (2013) is a standard reference for economists.
Anderson and Moore (1979) and Boyd (2009) are good technical references.

The starting point is the state-space system
\begin{eqnarray*}
    x_{t+1} &=& A x_t + C w_{t+1} \\
    y_t  &=&  G x_t + H v_t .
\end{eqnarray*}
Here $ \{w_t \}$ and $\{ v_t \}$ are vectors of independent standard normals ---
independent element by element, with each other, and across time.
We refer to $x_t$ as the state and $y_t$ as the measurement.
The state has dimension $n$, the measurement dimension $p$.
In the technical literature,
it's common to use $w_t$ in place of $w_{t+1}$ in the first equation,
but since neither $x_{t+1}$ nor $w_{t+1}$ is ever observed,
it's a convention without content.


{\it Controllability and observability.\/}
We say $(A,C)$ is {\it controllable\/} if
\begin{eqnarray*}
    \mathcal{C} &=&
    \left[
    \begin{array}{cccc}
        C  & A C & \cdots & A^{n-1} C
    \end{array}
    \right]
\end{eqnarray*}
has rank $n$.
The word controllable is misleading in this context;
the idea is simply that $w_t$ generates variation across
all $n$ dimensions of $x_t$.
We say $(A,G)$ is {\it observable\/} if
\begin{eqnarray*}
    \mathcal{O} &=&
    \left[
    \begin{array}{c}
        G \\ G  A  \\ \vdots \\ G  A ^{n-1}
    \end{array}
    \right]
\end{eqnarray*}
has rank $n$.
The issue is similar to the previous one.
The idea is that the condition guarantees us that observing the history of $y_t$
is enough to come up with a full-rank estimate of $x_t$.


Controllability example.
Here's one with $x_t$ of dimension two and $w_t$ of dimension one that fails:
\begin{eqnarray*}
    A &=&
    \left[
    \begin{array}{cc}
        a_{11} & a_{12} \\ 0 & a_{22}
    \end{array}
    \right],
    \;\;
    C \;\;=\;\;
    \left[
    \begin{array}{c}
        c_{1} \\ 0
    \end{array}
    \right]
    \;\;\;\Rightarrow \;\;\;
    \mathcal{C} \;\;=\;\;
    \left[
    \begin{array}{cc}
        c_{1} & a_{11} c_1 \\ 0 & 0
    \end{array}
    \right] ,
\end{eqnarray*}
which has rank $1 < n=2 $.
Here the innovation $w_t$ never generates variation in $x_{2t}$,
so we don't span the whole two-dimensional state.
However, if $a_{21}$ is nonzero we get controllability,
because $w_t$ affects $x_{2t}$ with a one-period lag through its impact on $x_{1t}$.
A similar example is an AR(2) in companion form.

Observability example.  The logic is similar.
Suppose $x_t$ is $n$-dimensional and the $n$th column of $G$
consists of zeros.
There's no direct impact of the $n$th state variable on
the observations $y_t$.
In the bond-pricing literature,
this might be a case in which one of the state variables
doesn't appear in bond yields of any maturity.
Nevertheless, the $n$th state variable might be (indirectly)
observable if it feeds into other state variables:
if $a_{jn}$ is nonzero for some $j\neq n$.
Here's an example similar to our previous one:
\begin{eqnarray*}
    A &=&
    \left[
    \begin{array}{cc}
        a_{11} & a_{12} \\  0 & a_{22}
    \end{array}
    \right],
    \;\;
    G \;\;=\;\;
    \left[
    \begin{array}{cc}
        g_{1} &  0
    \end{array}
    \right]
    \;\;\;\Rightarrow \;\;\;
    \mathcal{O} \;\;=\;\;
    \left[
    \begin{array}{cc}
        g_{1} & 0 \\ a_{11} g_1 & a_{12} g_1
    \end{array}
    \right] ,
\end{eqnarray*}
which has rank two.
Since $a_{12} = 0$, the condition fails.

%[?? mention Collin-Dufresne and Goldstein (2002) and Duffee (2011)]

\begin{comment}
{\it Kalman filter.\/}
The idea is to estimate the state $x_t$ given
(infinite) measurement histories $ y^s = (y_s, y_{s-1}, \ldots)$.
One popular notation is to express conditional expectations of
$x_t$ by
\begin{eqnarray*}
    \widehat{x}_{t|s} &=& E (x_t | y^s) .
\end{eqnarray*}
We use a more compact notation that's also common,
\begin{eqnarray*}
    \widehat{x}_{t}   &=& E (x_t | y^{t}) \\
    \widehat{x}_{t}^{-} &=& E ( x_t | y^{t-1} ) .
\end{eqnarray*}

Etc.
\end{comment}


\section{Stochastic volatility}
\label{app:stoch-vol}

We consider a representative agent model
with stochastic volatility:
the conditional variance is itself a stochastic process
and serves as an extra state variable.

The model is similar to the one in Section \ref{sec:rep-agent},
with these changes.
The law of motion for $x_t$ is now
\begin{eqnarray*}
    x_{t+1} &=&  A  x_t + C_t w_{t+1} ,
\end{eqnarray*}
where
\begin{eqnarray*}
   C_t &=& C \sqrt{S_t}, \quad S_{ii t} = \alpha_i+\beta_i v_t, \quad S_{ijt}=0, \\
       v_t &=& (1-\nu) v + \nu v_{t-1} + \sigma_v \sqrt{v_{t-1}} \varepsilon_t^v, \quad \sigma_v^2/2<(1-\nu)v
\end{eqnarray*}
%[If $v_t$ is not observed, id requires $\alpha_i=1$ and $\sigma_v=1.$]
The state now includes $x_t$ and $v_t$.
Shocks are $s_{it}=d_i^{\top}x_t+d_{iv}v_t.$

We solve by guess and verify.
If we guess $\pi_t=b^{\top}x_t + b_v v_t$,
the pricing relation gives us
\begin{eqnarray*}
    i_t = \rho + \alpha g - a^{\top} C_t C_t^{\top} a/2 + a^{\top} x_t + \nu (b_v+\alpha d_{1v}) v_t + (b_v+\alpha d_{1v})^2 v_t/2,
\end{eqnarray*}
where $a$ is the same as before.
The interest rate is therefore $i_t=a^{\top}x_t+a_v v_t$,
with $a_v = (b_v + \alpha d_{1v})^2/2 $.
Equating it to the Taylor rule implies
\begin{eqnarray*}
     a^{\top} &=& \tau b^{\top} +d_2^{\top} \\
     a_v \;\;=\;\; - a^{\top} C B C^{\top} a/2 + \nu (b_v+\alpha d_{1v}) + (b_v+\alpha d_{1v})^2 /2 &=& \tau b_v + d_{2v},
\end{eqnarray*}
where $ B$ is a diagonal matrix with elements $\beta_i$.

With respect to identification this has the same structure
as the constant volatility version.
What's new is the presence of volatility as a state variable.
If we think monetary policy does not respond to volatility,
then we can use the restriction $d_{2v} = 0$ to achieve identification.
Normandin and Phaneuf (2004) do something similar.





%\end{document}

% References ******************************************************************
\pagebreak
\parskip=0.5\bigskipamount
\parindent=0.0in
%\baselineskip=\oldbaselineskip
\section*{References}

\paper{Anderson, Brian D.O. and John B. Moore, 1979,
    {\it Optimal Filtering\/}, Englewood Cliffs, NJ:  Prentice Hall.}

\paper{Anderson, Evan W., Lars Peter Hansen, Ellen R. McGrattan,
    and Thomas J. Sargent,  1996,
    ``Mechanics of forming and estimating dynamic linear economies,''
    in H.M. Amman, D.A. Kendrick, and J. Rust,
    {\it Handbook of Computational Economics, Volume I\/}, 171-252.}

\paper{Ang, Andrew, and Monika Piazzesi, 2003,
    ``A no-arbitrage vector autoregression of term structure dynamics
    with macroeconomic and latent variables,''
    {\it Journal of Monetary Economics\/} 50, 745-787.}

\paper{Bai, Jushan, and Peng Wang, 2012,
    ``Identification and estimation of dynamic factor models,''
    manuscript, April.}

\paper{Bernanke, Ben, Jean Boivin, and Piotr S. Eliasz, 2005,
    ``Measuring the effects of monetary policy:
    A factor-augmented vector autoregressive (FAVAR) approach,''
    {\it Quarterly Journal of Economics\/} 120, 387-422.}

%\paper{Bernanke, Ben, and Ilian Mihov, 1998,
%    ``Measuring monetary policy,''
%    {\it Quarterly Journal of Economics\/} 113, 869-902.}

\paper{Boyd, Stephen P., 2009,
    ``Lecture slides for EE363: Linear Dynamical Systems,''
    posted online at \url{http://www.stanford.edu/class/ee363/}.}

\paper{Canova, Fabio, and Luca Sala, 2009,
    ``Back to square one:  Identification issues in DSGE models,''
    {\it Journal of Monetary Economics\/} 56, 431-449.}

\paper{Carrillo, Julio A., Patrick Feve, and Julien Matheron, 2007,
    ``Monetary policy or persistent shocks: a DGSE analysis,''
    {\it International Journal of Central Banking\/} 3, 1-38.}

\paper{Chernov, Mikhail, and Philippe Mueller, 2012,
    ``The term structure of inflation expectations,''
    {\it Journal of Financial Economics\/}, 106, 367-394.}

\paper{Christiano, Lawrence, Martin Eichenbaum, and Charles Evans, 2005,
     ``Nominal rigidities and the dynamic effects of a shock to monetary policy,''
     {\it Journal of Political Economy\/} 113, 1-45.}

\paper{Chun, Albert, 2011,
     ``Expectations, bond yields, and monetary policy,''
     {\it Review of FInancial Studies\/}, 24, 208-247.}

\paper{Clarida, Richard, 2001,
    ``The empirics of monetary policy rules in open economies,''
    {\it International Journal of Finance and Economics\/} 6, 315-323.}

\paper{Clarida, Richard, Jordi Gali, and Mark Gertler, 1999,
    ``The science of monetary policy,''
    {\it Journal of Economic Literature\/} 37, 1661-1707.}

\paper{Cochrane, John H., 2011,
    ``Determinacy and identification with Taylor rules,''
    {\it Journal of Political Economy\/} 119, 565-615.}

%\paper{Cochrane, John H., and Monika Piazzesi, AER, ...}

%\paper{Collin-Dufresne, Pierre, and Robert S. Goldstein, 2002,
%    ``Do bonds span the fixed income markets? Theory and evidence for
%    unspanned stochastic volatility,''
%    {\it Journal of Finance\/} 57, 1685-1730.}
%
%\paper{Duffee, Gregory R., 2011,
%    ``Information in (and not in) the term structure,''
%    {\it Review of Financial Studies\/} 20, 1669-1706.}

%\paper{Epstein, Larry G., and Stanley E. Zin, 1989,
%    ``Substitution, risk aversion, and the temporal behavior of consumption
%    and asset returns: a theoretical framework,''
%    {\it Econometrica\/} 57, 937-969.}

\paper{Gali, Jordi, 2008,
    {\it Monetary Policy, Inflation, and the Business Cycle\/},
    Princeton, NJ:  Princeton University Press.}

\paper{Gali, Jordi, and Mark Gertler, 1999,
    ``Inflation dynamics: A structural econometric analysis,''
    {\it Journal of Monetary Economics\/} 44, 195-222.}

\paper{Gallmeyer, Michael, Burton Hollifield, and Stanley Zin, 2005,
    ``Taylor rules, McCallum rules and the
    term structure of interest rates,''
    {\it Journal of Monetary Economics\/} 52, 921-950.}

%\paper{Gallmeyer, Michael F., Burton Hollifield,
%        Francisco Palomino, and Stanley Zin, 2007,
%    ``Arbitrage-free bond pricing with dynamic macroeconomic models,''
%    {\it Federal Reserve Bank of St Louis Review\/} 89, 305-26.}

\paper{Hansen, Lars, and Thomas Sargent, 1980,
    ``Formulating and estimating dynamic linear rational expectations
    models,''
    {\it Journal of Economic Dynamics and Control\/} 2, 7-46.}

%\paper{Hansen, Lars, and Thomas Sargent, 1981,
%    ``Linear rational expectations models for dynamically interrelated variables,''
%    {\it Rational Expectations and Econometric Practice\/},
%    ed. R.E. Lucas and T.J. Sargent, Minneapolis:  University of Minneapolis Press.}

\paper{Hansen, Lars, and Thomas Sargent, 1991,
    ``Two difficulties in interpreting vector autoregressions,''
    {\it Rational Expectations Econometrics\/},
    ed. L.P. Hansen and T.J. Sargent,
    San Francisco:  Westview Press.}

\paper{Hansen, Lars, and Thomas Sargent, 2013,
    {\it Recursive Models of Dynamic Linear Economies\/},
    Princeton:  Princeton University Press.}

\paper{Joslin, Scott, Anh Le, and Kenneth Singleton, 2013,
    ``Gaussian macro-finance term structure models with lags,''
    {\it Journal of Financial Econometrics\/}, forthcoming.}

%\paper{Joslin, Scott, Marcel Priebsch, and Kenneth Singleton, 2010,
%    ``Risk premiums in dynamic term structure models with unspanned macro risks,''
%    manuscript, October.}

\paper{Joslin, Scott, Kenneth Singleton, and Haoxiang Zhu, 2011,
   ``A New Perspective on Gaussian Dynamic Term Structure Models,''
      {\it Review of Financial Studies\/}, 24, 926-970}

\paper{Kim, Don, and Athanasios Orphanides, 2012,
    ``Term structure estimation with survey data on interest rate forecasts,''
           {\it Journal of Financial and Quantitative Analysis\/}, 47, 241-271.}

\paper{King, Robert, 2000,
    ``The new IS-LM model:  Language, logic, and limits,''
    {\it Federal Reserve Bank of Richmond Economic Quarterly\/}, 45-103.}

\paper{Klein, Paul, 2000,
    ``Using the generalized Schur form to solve a multivariate
    linear rational expectations model,''
    {\it Journal of Economic Dynamics \& Control\/} 24, 1405-1423.}

\paper{Leeper, Eric, Christopher Sims, and Tao Zha, 1996,
    ``What does monetary policy do?''
    {\it Brookings Papers on Economic Activity\/}, 1-63.}

%\paper{Lippi, Marco, and Lucrezia Reichlin, 1994,
%    ``VAR analysis, nonfundamental representations,
%    and Blaschke matrices,''
%    {\it Journal of Econometrics\/} 63, 307-325.}

\paper{Moench, Emanuel, 2008,
    ``Forecasting the yield curve in a data-rich environment:
    A no-arbitrage factor augmented VAR approach,''
    {\it Journal of Econometrics\/} 146, 26-43.}

\paper{Nason, James M., and Gregor W. Smith, 2008,
    ``The New Keynesian Phillips curve:
    Lessons from single-equation econometric estimation,''
    {\it Federal Reserve Bank of Richmond Economic Quarterly\/},
    361-395.}

\paper{Normandin, Michel, and Louis Phaneuf, 2004,
    ``Monetary policy shocks: Testing identification conditions
    under time-varying volatility,''
    {\it Journal of Monetary Economics\/} 51, 1217-1243.}

%\paper{Onatski, Alexi, factors, Econometrica, 2011??}

%\paper{Piazzesi, Monika, and Martin Schneider, 2007,
%    ``Equilibrium yield curves,'' in Daron Acemoglu, Kenneth Rogoff, and Michael Woodford,
%    eds., {\it NBER Macroeconomics Annual 2006\/},
%    MIT Press:  Cambridge MA.}

%\paper{Piazzesi, Monika, and Martin Schneider, 2011,
%    ``Trend and cycle in bond premia,''
%    manuscript, January.}

%\paper{Rubio-Ramirez, Juan, Daniel Waggoner, and Tao Zha, 2010,
%    ``Structural vector autoregressions: Theory of identification
%    and algorithms for inference,''
%    {\it Review of Economic Studies\/} 77, 665-696.}

\paper{Rudebusch, Glenn, and Tao Wu, 2008,
    ``A macro-finance model of the term structure, monetary policy,
    and the economy,''
    {\it Economic Journal\/} 118, 906-926.}

%\paper{Sargent, Thomas, 1972,
%    ``Rational expectations and the term structure
%    of interest rates,''
%    {\it Journal of Money, Credit and Banking\/} 4, 74-97.}

%\paper{Sargent, Thomas J.,  1987,
%    {\it Macroeconomic Theory (2nd edition)\/},
%    New York:  Academic Press.}

\paper{Shapiro, Adam Hale, 2008,
    ``Estimating the New Keynesian Phillips curve: A vertical
    production approach,''
    {\it Journal of Money, Credit and Banking\/} 40, 627-666.}

%\paper{Sims, Christopher, 1998,
%    ``Comment on Glenn Rudebusch's `Do measures of monetary policy
%    in a VAR make sense?'"
%    {\it International Economic Review\/} 39, 933-941.}

\paper{Sims, Christopher, and Tao Zha, 2006,
    ``Were there regime switches in US monetary policy?''
    {\it American Economic Review\/} 96, 54-81.}

\paper{Smith, Josephine, and John Taylor, 2009,
    ``The term structure of policy rules,''
    {\it Journal of Monetary Economics\/} 56, 907-917.}

%\paper{Stock, James H., and Mark W. Watson, 2013,
%    ``Disentangling the effects of the 2007-09 recession,''
%    {\it Brookings Papers on Economic Activity\/}, forthcoming.}

\paper{Watson, Mark, 1994,
    ``Vector autoregressions and cointegration,''
    in {\it Handbook of Econometrics, Volume IV\/},
    ed. R.F. Engel and D.L. McFadden,
    Elsevier.}

\paper{Woodford, Michael, 2003,
    {\it Interest and Prices\/}, Princeton, NJ: Princeton University Press.}

\end{document}

Any rule must be a restriction on "monetary policy responds in an arbitrary way
to every variable in the economy."

From Mike, Aug 1, 2013
how about the following:

instead of writing down m as in (5) or 3.2,

write it as m_{t+1}^{\$}=\lambda^{\top} x_{t+1}

with a proviso that is not identical, but similar to the pricing kernel used in the term structure literature for the Vasicek model. [The dynamics are similar
m_{t+1}^{\$}=\lambda^{\top} A x_t + \lambda^{\top} C w_{t+1}].

Then in section 2.2, i_t = - \lambda^{\top} A x_t (ignoring means)
and you may talk about - \lambda^{\top} A instead of delta

In section 3.2  you may talk about \lambda^{\top} A as d_1 and you get i_t = (b^{\top} - \lambda^{\top}) A

I think this a straightforward way to tighten the link between the exclusion restrictions and the affine models.

% ****************************************************************************
Old abstract

Identification problems arise naturally in forward-looking
models when agents observe more than econometricians.
One approach is to give the econometrician more information,
including economic forecasts and asset prices.
Another is tighter economic structure:
what is sometimes called cross-equation restrictions.
We show how each aids in the identification of structural
parameters, including the inflation parameter of a Taylor rule.
As a rule, asset prices help to identify the state,
and cross-equation restrictions help to identify structural parameters.

From Hansen-Sargent, 1991, ch 3, p 46:

In Hansen and Sargent (1980, JMCB), ... disturbances in decision rules
can have a variety of sources.
Disturbance terms can be interpreted as reflecting shocks ...
observed by private agents but not by the econometrician.
Disturbances can also be interpreted as reflecting interactions
with {\it hidden\/} decision variables which are simultaneously
chosen by private agents but unobserved by the econometrician.
Finally, disturbances can be interpreted ... as reflecting the
phenomenon that in forecasting the future, private agents use larger
information sets than the econometrician.

Asker:  ``A model is a measurement device, it gives us an interpretation of the data.''

Stan:
``If a rule depends on everything in an arbitrary way, it's not a rule.''



% **********************************************************************

Email paper to:
*Cochrane
*Gertler
*Peng Wang
*Clarida
*Zha
*Josie
*Moench
*Eichenbaum
Jesus
Sims
*Gali
*Nason
*Gregor

For Mike:
Monika
Amir
Woodford
Duffee
Singleton
Rudebusch
Swanson
Lars
